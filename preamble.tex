% This is my favourite config for bibliography. Each one of the references can be briefly explained with the field "note".
\RequirePackage[style=spanish]{csquotes}
\RequirePackage[
backend=biber,
style=numeric,
sorting=ynt
]{biblatex}
\newcommand{\familynameformat}[1]{\MakeUppercase{#1}}
\AtBeginBibliography{%
  \renewcommand{\mkbibnamefamily}{\familynameformat}%
}
\renewbibmacro*{begentry}{%
  \iffieldundef{note}
    {\undef\bbxnote}
    {\savefield{note}{\bbxnote}%
     \clearfield{note}}}
\renewbibmacro*{finentry}{%
  \restorefield{note}{\bbxnote}%
  \iffieldundef{note}
    {\finentry}
    {\setunit{\finentrypunct\par\vspace{\bibitemsep}\nobreak}
     \textit{\printfield{note}}%
     \finentry}}
\let\familynameformat=\textsc
\nocite{*}

\addbibresource{bibl.bib}

\pdfvariable minorversion 7

% Hyperref always required last one.
\RequirePackage{hyperref}
\makeatletter
\hypersetup{%
  colorlinks=true,
  linkcolor=magenta,
  urlcolor=cyan,
  citecolor=orange,
  pdfstartview=Fit,%
  pdfmenubar=true,%
  pdftoolbar=true,%
  bookmarksopen=false,%
  pdftitle={\@docsubtitle},%
  pdfauthor={\templateauthor},%
  pdfsubject={\@doctitle},%
  pdflang={\languagename},%
  pdfkeywords={mathematics},%
  pdfproducer={pdfTeX}}
\makeatother

% Setting up the titlepage.
\makeatletter%
%\title{{\large\textit{\@doctitlez}}\\[0.5cm]{\Huge\color{gray}\textsc{\@docsubtitle}}}%
\makeatother%
\date{\today}

% Paquetes útiles
\usepackage{multicol}

\usepackage{amsmath}
\usepackage{bm}

\usepackage[section]{placeins}

\usepackage{pdfpages}

\usepackage{siunitx}
\usepackage{physics}

\usepackage{tikz}
\usepackage{pgfplots}
\usepackage[dvipsnames]{xcolor}
\usepackage{xcolor}

% Configuración de Tikz/PGF
\usetikzlibrary{plotmarks}
\usetikzlibrary{arrows.meta}
\usetikzlibrary{babel}

% \usetikzlibrary{external}
% \tikzexternalize[prefix=.cache/tikz/, optimize command away=\includepdf]
 
\pgfplotsset{compat=newest}

% Otros ajustes
\DeclareSIUnit\noop{\relax}
\sisetup{group-separator = {,}, group-minimum-digits = 4}

% Columna centrada con ancho determinado para tablas
\newcolumntype{x}[1]{>{\centering\arraybackslash}m{#1}}
