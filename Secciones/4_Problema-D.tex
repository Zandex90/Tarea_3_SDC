\section{Pregunta \texttt{d)}}\label{pregunta-d}

\subsection{Desarrollo}

Luego, se grafica nuevamente el Nyquist (\autoref{fig:nyquist-d}) y Bode (\autoref{fig:bode-d})
tal como se hizo anteriormente, pero con ganancia de controlador $\mora{k_{cr}}$.

Se observa que para este caso, el margen de fase que se calcula utilizando
\textit{MATLAB} es
\begin{equation}
  \boxed{\textbf{M.F.} = \ang{-0.0034}}
\end{equation}

\begin{figure}[h]
  \centering
  % This file was created by matlab2tikz.
%
%The latest updates can be retrieved from
%  http://www.mathworks.com/matlabcentral/fileexchange/22022-matlab2tikz-matlab2tikz
%where you can also make suggestions and rate matlab2tikz.
%
\definecolor{mycolor1}{rgb}{0.00000,0.44700,0.74100}%
\definecolor{mycolor2}{rgb}{1.00000,0.27059,0.22745}%
\definecolor{mycolor3}{rgb}{0.49020,0.49020,0.49020}%
\definecolor{mycolor4}{rgb}{0.38039,0.38039,0.38039}%
%
\begin{tikzpicture}

\begin{axis}[%
width=9.785cm,
height=6cm,
at={(0cm,0cm)},
scale only axis,
unbounded coords=jump,
separate axis lines,
every outer x axis line/.append style={mycolor4},
every x tick label/.append style={font=\color{mycolor4}},
every x tick/.append style={mycolor4},
xmin=-4,
xmax=3,
every outer y axis line/.append style={mycolor4},
every y tick label/.append style={font=\color{mycolor4}},
every y tick/.append style={mycolor4},
ymin=-8,
ymax=8,
ylabel style={font=\color{mycolor4}},
ylabel={To: Out(1)},
axis background/.style={fill=white},
title style={font=\color{mycolor4}},
title={From: In(1)}
]
\addplot[only marks, mark=+, mark options={}, mark size=2.5000pt, draw=mycolor2, forget plot] table[]{Diagramas/data/nyquist_d-1.tsv};
\addplot [color=mycolor3, dotted, forget plot]
  table[]{Diagramas/data/nyquist_d-2.tsv};
\addplot [color=mycolor3, dotted, forget plot]
  table[]{Diagramas/data/nyquist_d-3.tsv};
\draw[dashdotted, draw=mycolor3] (axis cs:0,0) circle [radius=1];
\addplot [color=mycolor3, dashdotted, forget plot]
  table[]{Diagramas/data/nyquist_d-4.tsv};
\addplot [color=mycolor3, dashdotted, forget plot]
  table[]{Diagramas/data/nyquist_d-5.tsv};

\addplot[area legend, draw=mycolor1, fill=mycolor1, forget plot]
table[] {Diagramas/data/nyquist_d-6.tsv}--cycle;

\addplot[area legend, draw=mycolor1, fill=mycolor1, forget plot]
table[] {Diagramas/data/nyquist_d-7.tsv}--cycle;
\addplot [color=mycolor1, forget plot]
  table[]{Diagramas/data/nyquist_d-8.tsv};
\addplot[only marks, mark=*, mark options={}, mark size=1.5000pt, draw=mycolor1, fill=mycolor1, forget plot] table[]{Diagramas/data/nyquist_d-9.tsv};
\addplot[only marks, mark=*, mark options={}, mark size=1.5000pt, draw=mycolor1, fill=mycolor1, forget plot] table[]{Diagramas/data/nyquist_d-10.tsv};
\end{axis}
\end{tikzpicture}%

  \caption{Diagrama de Nyquist del sistema con ganancia crítica.}
  \label{fig:nyquist-d}
\end{figure}

\begin{figure}[h]
  \centering
  % This file was created by matlab2tikz.
%
%The latest updates can be retrieved from
%  http://www.mathworks.com/matlabcentral/fileexchange/22022-matlab2tikz-matlab2tikz
%where you can also make suggestions and rate matlab2tikz.
%
\definecolor{mycolor1}{rgb}{0.49020,0.49020,0.49020}%
\definecolor{mycolor2}{rgb}{0.00000,0.44700,0.74100}%
\definecolor{mycolor3}{rgb}{0.38039,0.38039,0.38039}%
\definecolor{mycolor4}{rgb}{0.12941,0.12941,0.12941}%
%
\begin{tikzpicture}

\begin{axis}[%
width=9.535cm,
height=2.765cm,
at={(0cm,0cm)},
scale only axis,
unbounded coords=jump,
separate axis lines,
every outer x axis line/.append style={mycolor3},
every x tick label/.append style={font=\color{mycolor3}},
every x tick/.append style={mycolor3},
xmode=log,
xmin=0.1,
xmax=100,
xminorticks=true,
every outer y axis line/.append style={mycolor3},
every y tick label/.append style={font=\color{mycolor3}},
every y tick/.append style={mycolor3},
ymin=-190,
ymax=-0,
ytick={-225, -180, -135,  -90,  -45,    0},
ylabel style={font=\color{mycolor4}},
ylabel={Phase (deg)},
axis background/.style={fill=white}
]
\addplot [color=mycolor1, line width=1.5pt, forget plot]
  table[]{Diagramas/data/bode_d-1.tsv};
\addplot [color=mycolor1, line width=1.5pt, forget plot]
  table[]{Diagramas/data/bode_d-2.tsv};
\addplot [color=mycolor1, dashdotted, forget plot]
  table[]{Diagramas/data/bode_d-3.tsv};
\addplot [color=mycolor1, dashdotted, forget plot]
  table[]{Diagramas/data/bode_d-4.tsv};
\addplot [color=mycolor2, forget plot]
  table[]{Diagramas/data/bode_d-5.tsv};
\addplot[only marks, mark=*, mark options={}, mark size=1.5000pt, draw=mycolor2, fill=white, forget plot] table[]{Diagramas/data/bode_d-6.tsv};
\addplot[only marks, mark=*, mark options={}, mark size=1.5000pt, draw=mycolor2, fill=mycolor2, forget plot] table[]{Diagramas/data/bode_d-7.tsv};
\addplot [color=mycolor1, dotted, forget plot]
  table[]{Diagramas/data/bode_d-8.tsv};
\addplot [color=mycolor1, dotted, forget plot]
  table[]{Diagramas/data/bode_d-9.tsv};
\end{axis}

\begin{axis}[%
width=9.535cm,
height=2.765cm,
at={(0cm,3.235cm)},
scale only axis,
unbounded coords=jump,
separate axis lines,
every outer x axis line/.append style={mycolor3},
every x tick label/.append style={font=\color{mycolor3}},
every x tick/.append style={mycolor3},
xmode=log,
xmin=0.1,
xmax=100,
xtick={0.1,1,10,100},
xticklabels={\empty},
xminorticks=true,
every outer y axis line/.append style={mycolor3},
every y tick label/.append style={font=\color{mycolor3}},
every y tick/.append style={mycolor3},
ymin=-50,
ymax=20,
ylabel style={font=\color{mycolor4}},
ylabel={Magnitude (dB)},
axis background/.style={fill=white},
title style={font=\color{mycolor3}},
title={From: In(1)}
]
\addplot [color=mycolor1, line width=1.5pt, forget plot]
  table[]{Diagramas/data/bode_d-10.tsv};
\addplot [color=mycolor1, line width=1.5pt, forget plot]
  table[]{Diagramas/data/bode_d-11.tsv};
\addplot [color=mycolor1, dashdotted, forget plot]
  table[]{Diagramas/data/bode_d-12.tsv};
\addplot [color=mycolor1, dashdotted, forget plot]
  table[]{Diagramas/data/bode_d-13.tsv};
\addplot [color=mycolor2, forget plot]
  table[]{Diagramas/data/bode_d-14.tsv};
\addplot[only marks, mark=*, mark options={}, mark size=1.5000pt, draw=mycolor2, fill=white, forget plot] table[]{Diagramas/data/bode_d-15.tsv};
\addplot[only marks, mark=*, mark options={}, mark size=1.5000pt, draw=mycolor2, fill=mycolor2, forget plot] table[]{Diagramas/data/bode_d-16.tsv};
\end{axis}
\end{tikzpicture}%

  \caption{Diagrama de Bode del sistema con ganancia crítica.}
  \label{fig:bode-d}
\end{figure}


\FloatBarrier
\subsection{Comentarios}

Si comenzamos analizando el diagrama de Nyquist, podemos observar que, al utilizar la ganancia crítica (con la cual realizamos el diagrama), la curva se expande en comparación con la vista en \hyperref[pregunta-b]{\texttt{b)}}. Esto ocurre hasta el punto en que el cruce por el eje real llega al punto crítico 
\((-1, 0)\).
Ahora viendo el M.F vemos que este tiende a cero, esto quiere decir que estamos al limite de la estabilidad, en este caso, cualquier perturbación o variación adicional en los parámetros del sistema puede llevarlo a la inestabilidad. En este punto, podemos afirmar que el sistema es marginalmente estable, por lo que oscilará indefinidamente, tal como se observó en la Tarea 2 \cite{tarea-2-sdc}.

Ademas el tiempo de retardo critico para este sistema sera igual a cero ya que se calcula con el M.F , por lo que el sistema no puede tolerar retardos significativos antes de volverse inestable.
