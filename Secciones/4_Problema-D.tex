\section{Pregunta \texttt{d)}}\label{pregunta-d}

\subsection{Desarrollo}

Luego, se grafica nuevamente el Nyquist (\autoref{fig:nyquist-d}) y Bode (\autoref{fig:bode-d})
tal como se hizo anteriormente, pero con ganancia de controlador $\mora{k_{cr}}$.

Se observa que para este caso, el margen de fase que se calcula utilizando
\textit{MATLAB} es
\begin{equation}
  \boxed{\textbf{M.F.} = \ang{-0.0034}}
\end{equation}

\begin{figure}[h]
  \centering
  % This file was created by matlab2tikz.
%
%The latest updates can be retrieved from
%  http://www.mathworks.com/matlabcentral/fileexchange/22022-matlab2tikz-matlab2tikz
%where you can also make suggestions and rate matlab2tikz.
%
\begin{tikzpicture}

\begin{axis}[%
width=9.785cm,
height=6cm,
at={(0cm,0cm)},
scale only axis,
unbounded coords=jump,
separate axis lines,
every outer x axis line/.append style={white!15!black},
every x tick label/.append style={font=\color{white!15!black}},
every x tick/.append style={white!15!black},
xmin=-4,
xmax=3,
every outer y axis line/.append style={white!15!black},
every y tick label/.append style={font=\color{white!15!black}},
every y tick/.append style={white!15!black},
ymin=-8,
ymax=8,
ylabel style={font=\color{white!15!black}},
ylabel={Eje imaginario},
xlabel={Eje real},
axis background/.style={fill=white},
legend cell align=left,
]
\addplot[only marks, mark=+, mark options={}, mark size=2.5000pt, draw=red, forget plot] table[]{Diagramas/data/nyquist_d-1.tsv};
\addplot [color=gray, dotted, forget plot]
  table[]{Diagramas/data/nyquist_d-2.tsv};
\addplot [color=gray, dotted, forget plot]
  table[]{Diagramas/data/nyquist_d-3.tsv};
\draw[dashdotted, draw=gray] (axis cs:0,0) circle [radius=1];
\addplot [color=gray, dashdotted, forget plot]
  table[]{Diagramas/data/nyquist_d-4.tsv};
\addplot [color=gray, dashdotted, forget plot]
  table[]{Diagramas/data/nyquist_d-5.tsv};

\addplot[area legend, draw=mycolor2, fill=mycolor2, forget plot]
table[] {Diagramas/data/nyquist_d-6.tsv}--cycle;

\addplot[area legend, draw=mycolor2, fill=mycolor2, forget plot]
table[] {Diagramas/data/nyquist_d-7.tsv}--cycle;
\addplot [color=mycolor2, forget plot]
  table[]{Diagramas/data/nyquist_d-8.tsv};
\addplot[only marks, mark=*, mark options={}, mark size=1.5000pt, draw=mycolor1, fill=mycolor1] table[]{Diagramas/data/nyquist_d-9.tsv};
  \addlegendentry{Margen de ganancia}
\addplot[only marks, mark=*, mark options={}, mark size=1.5000pt, draw=mycolor4, fill=mycolor4] table[]{Diagramas/data/nyquist_d-10.tsv};
  \addlegendentry{Margen de fase}
\end{axis}
\end{tikzpicture}%

  \caption{Diagrama de Nyquist del sistema con ganancia crítica.}
  \label{fig:nyquist-d}
\end{figure}

\begin{figure}[h]
  \centering
  % This file was created by matlab2tikz.
%
%The latest updates can be retrieved from
%  http://www.mathworks.com/matlabcentral/fileexchange/22022-matlab2tikz-matlab2tikz
%where you can also make suggestions and rate matlab2tikz.
%
\begin{tikzpicture}

\begin{axis}[%
width=9.535cm,
height=2.765cm,
at={(0cm,0cm)},
scale only axis,
unbounded coords=jump,
separate axis lines,
every outer x axis line/.append style={white!15!black},
every x tick label/.append style={font=\color{white!15!black}},
every x tick/.append style={white!15!black},
xmode=log,
xmin=0.1,
xmax=100,
xminorticks=true,
every outer y axis line/.append style={white!15!black},
every y tick label/.append style={font=\color{white!15!black}},
every y tick/.append style={white!15!black},
ymin=-190,
ymax=-0,
ytick={-225, -180, -135,  -90,  -45,    0},
ylabel style={font=\color{white!15!black}},
ylabel={Fase $[\unit{deg}]$},
axis background/.style={fill=white}
]
\addplot [color=gray, line width=1.5pt, forget plot]
  table[]{Diagramas/data/bode_d-1.tsv};
\addplot [color=gray, line width=1.5pt, forget plot]
  table[]{Diagramas/data/bode_d-2.tsv};
\addplot [color=gray, dashdotted, forget plot]
  table[]{Diagramas/data/bode_d-3.tsv};
\addplot [color=gray, dashdotted, forget plot]
  table[]{Diagramas/data/bode_d-4.tsv};
\addplot [color=mycolor2, forget plot]
  table[]{Diagramas/data/bode_d-5.tsv};
\addplot[only marks, mark=*, mark options={}, mark size=1.5000pt, draw=mycolor2, fill=white, forget plot] table[]{Diagramas/data/bode_d-6.tsv};
\addplot[only marks, mark=*, mark options={}, mark size=1.5000pt, draw=mycolor2, fill=mycolor2, forget plot] table[]{Diagramas/data/bode_d-7.tsv};
\addplot [color=gray, dotted, forget plot]
  table[]{Diagramas/data/bode_d-8.tsv};
\addplot [color=gray, dotted, forget plot]
  table[]{Diagramas/data/bode_d-9.tsv};
\end{axis}

\begin{axis}[%
width=9.535cm,
height=2.765cm,
at={(0cm,3.235cm)},
scale only axis,
unbounded coords=jump,
separate axis lines,
every outer x axis line/.append style={white!15!black},
every x tick label/.append style={font=\color{white!15!black}},
every x tick/.append style={white!15!black},
xmode=log,
xmin=0.1,
xmax=100,
xtick={0.1,1,10,100},
xticklabels={\empty},
xminorticks=true,
every outer y axis line/.append style={white!15!black},
every y tick label/.append style={font=\color{white!15!black}},
every y tick/.append style={white!15!black},
ymin=-50,
ymax=20,
ylabel style={font=\color{white!15!black}},
ylabel={Magnitud $[\unit{dB}]$},
axis background/.style={fill=white},
title style={font=\color{white!15!black}},
]
\addplot [color=gray, line width=1.5pt, forget plot]
  table[]{Diagramas/data/bode_d-10.tsv};
\addplot [color=gray, line width=1.5pt, forget plot]
  table[]{Diagramas/data/bode_d-11.tsv};
\addplot [color=gray, dashdotted, forget plot]
  table[]{Diagramas/data/bode_d-12.tsv};
\addplot [color=gray, dashdotted, forget plot]
  table[]{Diagramas/data/bode_d-13.tsv};
\addplot [color=mycolor2, forget plot]
  table[]{Diagramas/data/bode_d-14.tsv};
\addplot[only marks, mark=*, mark options={}, mark size=1.5000pt, draw=mycolor2, fill=white, forget plot] table[]{Diagramas/data/bode_d-15.tsv};
\addplot[only marks, mark=*, mark options={}, mark size=1.5000pt, draw=mycolor2, fill=mycolor2, forget plot] table[]{Diagramas/data/bode_d-16.tsv};
\end{axis}
\end{tikzpicture}%

  \caption{Diagrama de Bode del sistema con ganancia crítica.}
  \label{fig:bode-d}
\end{figure}


\FloatBarrier
\subsection{Comentarios}

\textbf{NO OLVIDAR!!!}



%%% D %%%
\begin{itemize}
  \item Si existe la ganancia crítica qué característica tiene el Nyquist.
  \item Si existe la ganancia crítica qué característica tiene el Bode.
  \item Porqué el MF vale cero.
  \item Porqué el sistema queda oscilando indefinidamente.
  \item Cuál es la frecuencia de esta oscilación.
  \item Qué parámetro define la frecuencia de esta oscilación.
\end{itemize}
