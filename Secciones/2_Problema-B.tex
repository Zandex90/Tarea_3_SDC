\section{Pregunta \texttt{b)}}\label{pregunta-b}

\subsection{Desarrollo}

A continuación, se graficará los diagramas de Nyquist (\autoref{fig:nyquist-b})
y de Bode (\autoref{fig:bode-b}) del sistema con las ganancias dadas en la pregunta
anterior. Para esto, se utilizó los comandos \texttt{nyquistplot} y \texttt{bodeplot}
respectivamente, los cuales permiten ser configurados para que el M.F. y M.G. sean
identificados en las gráficas (\autoref{lst:prob-b}).

\begin{figure}[h]
  \centering
  % This file was created by matlab2tikz.
%
%The latest updates can be retrieved from
%  http://www.mathworks.com/matlabcentral/fileexchange/22022-matlab2tikz-matlab2tikz
%where you can also make suggestions and rate matlab2tikz.
%
\begin{tikzpicture}

\begin{axis}[%
width=7.339cm,
height=6cm,
at={(0cm,0cm)},
scale only axis,
unbounded coords=jump,
separate axis lines,
every outer x axis line/.append style={white!15!black},
every x tick label/.append style={font=\color{white!15!black}},
every x tick/.append style={white!15!black},
xmin=-2,
xmax=1.5,
every outer y axis line/.append style={white!15!black},
every y tick label/.append style={font=\color{white!15!black}},
every y tick/.append style={white!15!black},
ymin=-3,
ymax=3,
ylabel style={font=\color{white!15!black}},
ylabel={Eje imaginario},
xlabel style={font=\color{white!15!black}},
xlabel={Eje real},
axis background/.style={fill=white},
legend cell align=left,
]
\draw[dashdotted, draw=gray] (axis cs:0,0) circle [radius=1];

\addplot [color=gray, dotted, forget plot]
  table[]{Diagramas/data/nyquist_b-2.tsv};
\addplot [color=gray, dotted, forget plot]
  table[]{Diagramas/data/nyquist_b-3.tsv};

\addplot [color=gray, dashdotted, forget plot]
  table[]{Diagramas/data/nyquist_b-4.tsv};
\addplot [color=gray, dashdotted, forget plot]
  table[]{Diagramas/data/nyquist_b-5.tsv};

\addplot [color=mycolor2, forget plot]
  table[]{Diagramas/data/nyquist_b-8.tsv};

\addplot[area legend, draw=mycolor2, fill=mycolor2, forget plot]
  table[] {Diagramas/data/nyquist_b-6.tsv}--cycle;
\addplot[area legend, draw=mycolor2, fill=mycolor2, forget plot]
  table[] {Diagramas/data/nyquist_b-7.tsv}--cycle;

\addplot[only marks, mark=+, mark options={}, mark size=2.5000pt,
  draw=red, forget plot] table[]{Diagramas/data/nyquist_b-1.tsv};
\addplot[only marks, mark=*, mark options={}, mark size=1.5000pt,
  draw=mycolor1, fill=mycolor1] table[]{Diagramas/data/nyquist_b-9.tsv};
\addlegendentry{Margen de ganancia}
\addplot[only marks, mark=*, mark options={}, mark size=1.5000pt,
  draw=mycolor4, fill=mycolor4] table[]{Diagramas/data/nyquist_b-10.tsv};
\addlegendentry{Margen de fase}
\end{axis}
\end{tikzpicture}%

  \caption{Diagrama de Nyquist del sistema con $\mora{k_{c}} = 10\unit{m}$.}
  \label{fig:nyquist-b}
\end{figure}

\begin{figure}[h]
  \centering
  % This file was created by matlab2tikz.
%
%The latest updates can be retrieved from
%  http://www.mathworks.com/matlabcentral/fileexchange/22022-matlab2tikz-matlab2tikz
%where you can also make suggestions and rate matlab2tikz.
%
\begin{tikzpicture}

\begin{axis}[%
width=9.535cm,
height=2.765cm,
at={(0cm,0cm)},
scale only axis,
unbounded coords=jump,
separate axis lines,
every outer x axis line/.append style={white!15!black},
every x tick label/.append style={font=\color{white!15!black}},
every x tick/.append style={white!15!black},
xmode=log,
xmin=0.1,
xmax=100,
xminorticks=true,
every outer y axis line/.append style={white!15!black},
every y tick label/.append style={font=\color{white!15!black}},
every y tick/.append style={white!15!black},
ymin=-190,
ymax=-0,
ytick={-225, -180, -135,  -90,  -45,    0},
ylabel style={font=\color{white!15!black}},
ylabel={Fase $[\unit{deg}]$},
xlabel={Frecuencia $[\unit{rad/s}]$},
axis background/.style={fill=white}
]
\addplot [color=gray, line width=1.5pt, forget plot]
  table[]{Diagramas/data/bode_b-1.tsv};
\addplot [color=gray, line width=1.5pt, forget plot]
  table[]{Diagramas/data/bode_b-2.tsv};
\addplot [color=gray, dashdotted, forget plot]
  table[]{Diagramas/data/bode_b-3.tsv};
\addplot [color=gray, dashdotted, forget plot]
  table[]{Diagramas/data/bode_b-4.tsv};
\addplot [color=mycolor2, forget plot]
  table[]{Diagramas/data/bode_b-5.tsv};
\addplot[only marks, mark=*, mark options={}, mark size=1.5000pt, draw=mycolor2, fill=mycolor2, forget plot] table[]{Diagramas/data/bode_b-7.tsv};
\addplot [color=gray, dotted, forget plot]
  table[]{Diagramas/data/bode_b-8.tsv};
\addplot [color=gray, dotted, forget plot]
  table[]{Diagramas/data/bode_b-9.tsv};
\end{axis}

\begin{axis}[%
width=9.535cm,
height=2.765cm,
at={(0cm,3.235cm)},
scale only axis,
unbounded coords=jump,
separate axis lines,
every outer x axis line/.append style={white!15!black},
every x tick label/.append style={font=\color{white!15!black}},
every x tick/.append style={white!15!black},
xmode=log,
xmin=0.1,
xmax=100,
xtick={0.1,1,10,100},
xticklabels={\empty},
xminorticks=true,
every outer y axis line/.append style={white!15!black},
every y tick label/.append style={font=\color{white!15!black}},
every y tick/.append style={white!15!black},
ymin=-50,
ymax=10,
ylabel style={font=\color{white!15!black}},
ylabel={Magnitud $[\unit{dB}]$},
axis background/.style={fill=white},
]
\addplot [color=gray, line width=1.5pt, forget plot]
  table[]{Diagramas/data/bode_b-10.tsv};
\addplot [color=gray, line width=1.5pt, forget plot]
  table[]{Diagramas/data/bode_b-11.tsv};
\addplot [color=gray, dashdotted, forget plot]
  table[]{Diagramas/data/bode_b-12.tsv};
\addplot [color=gray, dashdotted, forget plot]
  table[]{Diagramas/data/bode_b-13.tsv};
\addplot [color=mycolor2, forget plot]
  table[]{Diagramas/data/bode_b-14.tsv};
\addplot[only marks, mark=*, mark options={}, mark size=1.5000pt,
  draw=mycolor2, fill=mycolor2, forget plot] table[]{Diagramas/data/bode_b-16.tsv};
\end{axis}
\end{tikzpicture}%

  \caption{Diagrama de Bode del sistema con $\mora{k_{c}} = 10\unit{m}$.}
  \label{fig:bode-b}
\end{figure}

\subsection{Comentarios}


Al observar el diagrama de Bode, podemos visualizar que, para el gráfico de magnitud, la curva sí presenta un cruce por el eje \( 0 \ \text{dB} \), indicando que existe una frecuencia de cruce de ganancia unitaria. Por otro lado, en el gráfico de fase, la curva tiende al valor de \(-180^\circ\), pero no lo cruza en la frecuencia de cruce de ganancia. Estas observaciones corroboran que el sistema presenta un margen de fase positivo, tal como fue calculado en la letra A, con un valor de aproximadamente \( 9.9833^\circ \).

En el caso del diagrama de Nyquist, podemos observar que la curva cruza el eje real en un punto cercano a \(-0.5\), pero no rodea el punto crítico \((-1, 0)\). Esto confirma la existencia de un M.G. positivo, ya que no hay un cruce o envolvimiento que indique una inestabilidad. Asimismo, la curva mantiene una distancia angular suficiente respecto al punto crítico, lo que confirma la existencia del margen de fase. Bajo estas observaciones, podemos comprobar que los resultados obtenidos en la letra  \hyperref[pregunta-c]{\texttt{a)}} están bien. En este caso asegurando estabilidad, solo que al limite de esta como se menciono en los comentarios del apartado \hyperref[pregunta-c]{\texttt{a)}}.

