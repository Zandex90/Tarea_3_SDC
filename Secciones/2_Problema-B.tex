\section{Pregunta \texttt{b)}}\label{pregunta-b}

\subsection{Desarrollo}

A continuación, se graficará los diagramas de Nyquist (\autoref{fig:nyquist-b})
y de Bode (\autoref{fig:bode-b}) del sistema con las ganancias dadas en la pregunta
anterior. Para esto, se utilizó los comandos \texttt{nyquistplot} y \texttt{bodeplot}
respectivamente, los cuales permiten ser configurados para que el M.F. y M.G. sean
identificados en las gráficas (\autoref{lst:prob-b}).

\begin{figure}[h]
  \centering
  % This file was created by matlab2tikz.
%
%The latest updates can be retrieved from
%  http://www.mathworks.com/matlabcentral/fileexchange/22022-matlab2tikz-matlab2tikz
%where you can also make suggestions and rate matlab2tikz.
%
\begin{tikzpicture}

\begin{axis}[%
width=7.339cm,
height=6cm,
at={(0cm,0cm)},
scale only axis,
unbounded coords=jump,
separate axis lines,
every outer x axis line/.append style={white!15!black},
every x tick label/.append style={font=\color{white!15!black}},
every x tick/.append style={white!15!black},
xmin=-2,
xmax=1.5,
every outer y axis line/.append style={white!15!black},
every y tick label/.append style={font=\color{white!15!black}},
every y tick/.append style={white!15!black},
ymin=-3,
ymax=3,
ylabel style={font=\color{white!15!black}},
ylabel={Eje imaginario},
xlabel style={font=\color{white!15!black}},
xlabel={Eje real},
axis background/.style={fill=white},
legend cell align=left,
]
\draw[dashdotted, draw=gray] (axis cs:0,0) circle [radius=1];

\addplot [color=gray, dotted, forget plot]
  table[]{Diagramas/data/nyquist_b-2.tsv};
\addplot [color=gray, dotted, forget plot]
  table[]{Diagramas/data/nyquist_b-3.tsv};

\addplot [color=gray, dashdotted, forget plot]
  table[]{Diagramas/data/nyquist_b-4.tsv};
\addplot [color=gray, dashdotted, forget plot]
  table[]{Diagramas/data/nyquist_b-5.tsv};

\addplot [color=mycolor2, forget plot]
  table[]{Diagramas/data/nyquist_b-8.tsv};

\addplot[area legend, draw=mycolor2, fill=mycolor2, forget plot]
  table[] {Diagramas/data/nyquist_b-6.tsv}--cycle;
\addplot[area legend, draw=mycolor2, fill=mycolor2, forget plot]
  table[] {Diagramas/data/nyquist_b-7.tsv}--cycle;

\addplot[only marks, mark=+, mark options={}, mark size=2.5000pt,
  draw=red, forget plot] table[]{Diagramas/data/nyquist_b-1.tsv};
\addplot[only marks, mark=*, mark options={}, mark size=1.5000pt,
  draw=mycolor1, fill=mycolor1] table[]{Diagramas/data/nyquist_b-9.tsv};
\addlegendentry{Margen de ganancia}
\addplot[only marks, mark=*, mark options={}, mark size=1.5000pt,
  draw=mycolor4, fill=mycolor4] table[]{Diagramas/data/nyquist_b-10.tsv};
\addlegendentry{Margen de fase}
\end{axis}
\end{tikzpicture}%

  \caption{Diagrama de Nyquist del sistema con $\mora{k_{c}} = 10\unit{m}$.}
  \label{fig:nyquist-b}
\end{figure}

\begin{figure}[h]
  \centering
  \input{Diagramas/bode_b.tex}
  \caption{Diagrama de Bode del sistema con $\mora{k_{c}} = 10\unit{m}$.}
  \label{fig:bode-b}
\end{figure}

\subsection{Comentarios}

\textbf{NO OLVIDAR!!!}

%%% B %%%
\begin{itemize}
  \item Si existe el MF qué característica tiene el Nyquist.
  \item Si no existe el MF qué característica tiene el Nyquist.
  \item Qué implica si no existe el MF.
  \item Si existe el MG qué característica tiene el Bode.
  \item Si no existe el MG qué característica tiene el Bode.
  \item Qué implica si no existe el MG.
\end{itemize}
