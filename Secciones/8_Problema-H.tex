\section{Pregunta \texttt{h)}}\label{pregunta-h}


\subsection{Desarrollo}

Nuevamente nos piden calcular el margen de ganancia (M.G.) y margen de fase
(M.F.). solo que para un controlador \(h_c(z)= \mora{k_c}/(z-1) = 4m / (z-1)\).
Entonces, sabiendo que la función de transferencia del sistema es,
\begin{equation}
  LD(z) = \frac{\mora{k_{c}}}{(z-1)} \nara{ k_{a}} z^{-1} \cdot \frac{\rojo{\psi}(z)}{\verd{v_i}(z)} \nara{k_{st}}  
\end{equation}

Se procede a calcular el margen de fase y margen de ganancia del sistema:
\begin{equation}
  \boxed{\textbf{M.G.} = 1.7348 = 4.7852\ \unit{dB}} \qquad \boxed{\textbf{M.F.} = \ang{22.5213}}
\end{equation}

Luego, se dibujan los gráficos de interés (\autoref{fig:nyquist-h1} y \autoref{fig:bode-h1}),
y se determina el valor de la ganancia y retardo críticos:
\begin{equation}
  \boxed{\mora{k_{cr}} = 6.9\unit{m}} \qquad \boxed{\mora{N_{cr}} = 1\ \unit{muestra}}
\end{equation}

Al igual que para el caso discreto anterior, el retardo crítico había dado
originalmente un valor con parte fraccionaria, por lo que se tuvo que redondear.

\begin{figure}[h]
  \centering
  % This file was created by matlab2tikz.
%
%The latest updates can be retrieved from
%  http://www.mathworks.com/matlabcentral/fileexchange/22022-matlab2tikz-matlab2tikz
%where you can also make suggestions and rate matlab2tikz.
%
\definecolor{mycolor1}{rgb}{0.00000,0.44700,0.74100}%
\definecolor{mycolor2}{rgb}{1.00000,0.27059,0.22745}%
\definecolor{mycolor3}{rgb}{0.49020,0.49020,0.49020}%
\definecolor{mycolor4}{rgb}{0.38039,0.38039,0.38039}%
%
\begin{tikzpicture}

\begin{axis}[%
width=9.785cm,
height=6cm,
at={(0cm,0cm)},
scale only axis,
unbounded coords=jump,
separate axis lines,
every outer x axis line/.append style={mycolor4},
every x tick label/.append style={font=\color{mycolor4}},
every x tick/.append style={mycolor4},
xmin=-1.6,
xmax=0.6,
every outer y axis line/.append style={mycolor4},
every y tick label/.append style={font=\color{mycolor4}},
every y tick/.append style={mycolor4},
ymin=-5,
ymax=5,
ylabel style={font=\color{mycolor4}},
ylabel={To: Out(1)},
axis background/.style={fill=white},
title style={font=\color{mycolor4}},
title={From: In(1)},
restrict y to domain=-20:20
]
\addplot[only marks, mark=+, mark options={}, mark size=2.5000pt, draw=mycolor2, forget plot] table[]{Diagramas/data/nyquist_h1-1.tsv};
\addplot [color=mycolor3, dotted, forget plot]
  table[]{Diagramas/data/nyquist_h1-2.tsv};
\addplot [color=mycolor3, dotted, forget plot]
  table[]{Diagramas/data/nyquist_h1-3.tsv};
\draw[dashdotted, draw=mycolor3] (axis cs:0,0) circle [radius=1];
\addplot [color=mycolor3, dashdotted, forget plot]
  table[]{Diagramas/data/nyquist_h1-4.tsv};
\addplot [color=mycolor3, dashdotted, forget plot]
  table[]{Diagramas/data/nyquist_h1-5.tsv};

\addplot[area legend, draw=mycolor1, fill=mycolor1, forget plot]
table[] {Diagramas/data/nyquist_h1-6.tsv}--cycle;

\addplot[area legend, draw=mycolor1, fill=mycolor1, forget plot]
table[] {Diagramas/data/nyquist_h1-7.tsv}--cycle;
\addplot [color=mycolor1, forget plot]
  table[]{Diagramas/data/nyquist_h1-8.tsv};
\addplot[only marks, mark=*, mark options={}, mark size=1.5000pt, draw=mycolor1, fill=mycolor1, forget plot] table[]{Diagramas/data/nyquist_h1-9.tsv};
\addplot[only marks, mark=*, mark options={}, mark size=1.5000pt, draw=mycolor1, fill=mycolor1, forget plot] table[]{Diagramas/data/nyquist_h1-10.tsv};
\end{axis}
\end{tikzpicture}%

  \caption{Diagrama de Nyquist del sistema con $h_{c}(z) = \mora{k_{c}}\mathbin{/}(z-1)$.}
  \label{fig:nyquist-h1}
\end{figure}

\begin{figure}[h]
  \centering
  % This file was created by matlab2tikz.
%
%The latest updates can be retrieved from
%  http://www.mathworks.com/matlabcentral/fileexchange/22022-matlab2tikz-matlab2tikz
%where you can also make suggestions and rate matlab2tikz.
%
\begin{tikzpicture}

\begin{axis}[%
width=9.632cm,
height=2.765cm,
at={(0cm,0cm)},
scale only axis,
unbounded coords=jump,
separate axis lines,
every outer x axis line/.append style={white!15!black},
every x tick label/.append style={font=\color{white!15!black}},
every x tick/.append style={white!15!black},
xmode=log,
xmin=0.01,
xmax=15.707963267949,
xminorticks=true,
every outer y axis line/.append style={white!15!black},
every y tick label/.append style={font=\color{white!15!black}},
every y tick/.append style={white!15!black},
ymin=-720,
ymax=-90,
ytick={-720, -540, -360, -180},
ylabel style={font=\color{white!15!black}},
ylabel={Fase $[\unit{deg}]$},
axis background/.style={fill=white}
]
\addplot [color=gray, line width=1.5pt, forget plot]
  table[]{Diagramas/data/bode_h1-1.tsv};
\addplot [color=gray, line width=1.5pt, forget plot]
  table[]{Diagramas/data/bode_h1-2.tsv};
\addplot [color=gray, dashdotted, forget plot]
  table[]{Diagramas/data/bode_h1-3.tsv};
\addplot [color=gray, dashdotted, forget plot]
  table[]{Diagramas/data/bode_h1-4.tsv};
\addplot [color=mycolor2, forget plot]
  table[]{Diagramas/data/bode_h1-5.tsv};
\addplot[only marks, mark=*, mark options={}, mark size=1.5000pt, draw=mycolor2, fill=white, forget plot] table[]{Diagramas/data/bode_h1-6.tsv};
\addplot[only marks, mark=*, mark options={}, mark size=1.5000pt, draw=mycolor2, fill=mycolor2, forget plot] table[]{Diagramas/data/bode_h1-7.tsv};
\end{axis}

\begin{axis}[%
width=9.632cm,
height=2.765cm,
at={(0cm,3.235cm)},
scale only axis,
unbounded coords=jump,
separate axis lines,
every outer x axis line/.append style={white!15!black},
every x tick label/.append style={font=\color{white!15!black}},
every x tick/.append style={white!15!black},
xmode=log,
xmin=0.01,
xmax=15.707963267949,
xtick={0.01,1,10},
xticklabels={\empty},
xminorticks=true,
every outer y axis line/.append style={white!15!black},
every y tick label/.append style={font=\color{white!15!black}},
every y tick/.append style={white!15!black},
ymin=-60,
ymax=40,
ylabel style={font=\color{white!15!black}},
ylabel={Magnitud $[\unit{dB}]$},
axis background/.style={fill=white},
title style={font=\color{white!15!black}},
]
\addplot [color=gray, line width=1.5pt, forget plot]
  table[]{Diagramas/data/bode_h1-8.tsv};
\addplot [color=gray, line width=1.5pt, forget plot]
  table[]{Diagramas/data/bode_h1-9.tsv};
\addplot [color=gray, dashdotted, forget plot]
  table[]{Diagramas/data/bode_h1-10.tsv};
\addplot [color=gray, dashdotted, forget plot]
  table[]{Diagramas/data/bode_h1-11.tsv};
\addplot [color=mycolor2, forget plot]
  table[]{Diagramas/data/bode_h1-12.tsv};
\addplot[only marks, mark=*, mark options={}, mark size=1.5000pt, draw=mycolor2, fill=white, forget plot] table[]{Diagramas/data/bode_h1-13.tsv};
\addplot[only marks, mark=*, mark options={}, mark size=1.5000pt, draw=mycolor2, fill=mycolor2, forget plot] table[]{Diagramas/data/bode_h1-14.tsv};
\end{axis}
\end{tikzpicture}%

  \caption{Diagrama de Bode del sistema con $h_{c}(z) = \mora{k_{c}}\mathbin{/}(z-1)$.}
  \label{fig:bode-h1}
\end{figure}

Con esto, se pudo graficar para el sistema con ganancia crítica (\autoref{fig:nyquist-h2} y
\autoref{fig:bode-h2}). Donde se observó que se tiene un margen de fase de:
\begin{equation}
  \boxed{\textbf{M.F.} = \ang{-0.00024164}}
\end{equation}

\begin{figure}[h]
  \centering
  % This file was created by matlab2tikz.
%
%The latest updates can be retrieved from
%  http://www.mathworks.com/matlabcentral/fileexchange/22022-matlab2tikz-matlab2tikz
%where you can also make suggestions and rate matlab2tikz.
%
\begin{tikzpicture}

\begin{axis}[%
width=9.785cm,
height=6cm,
at={(0cm,0cm)},
scale only axis,
unbounded coords=jump,
separate axis lines,
every outer x axis line/.append style={white!15!black},
every x tick label/.append style={font=\color{white!15!black}},
every x tick/.append style={white!15!black},
xmin=-1.2,
xmax=1.2,
every outer y axis line/.append style={white!15!black},
every y tick label/.append style={font=\color{white!15!black}},
every y tick/.append style={white!15!black},
ymin=-2,
ymax=2,
ylabel style={font=\color{white!15!black}},
ylabel={Eje imaginario},
xlabel style={font=\color{white!15!black}},
xlabel={Eje real},
axis background/.style={fill=white},
legend cell align=left,
restrict y to domain=-20:20
]
\addplot[only marks, mark=+, mark options={}, mark size=2.5000pt, draw=red, forget plot] table[]{Diagramas/data/nyquist_h2-1.tsv};
\addplot [color=gray, dotted, forget plot]
  table[]{Diagramas/data/nyquist_h2-2.tsv};
\addplot [color=gray, dotted, forget plot]
  table[]{Diagramas/data/nyquist_h2-3.tsv};
\draw[dashdotted, draw=gray] (axis cs:0,0) circle [radius=1];
\addplot [color=gray, dashdotted, forget plot]
  table[]{Diagramas/data/nyquist_h2-4.tsv};
\addplot [color=gray, dashdotted, forget plot]
  table[]{Diagramas/data/nyquist_h2-5.tsv};

\addplot[area legend, draw=mycolor2, fill=mycolor2, forget plot]
table[] {Diagramas/data/nyquist_h2-6.tsv}--cycle;

\addplot[area legend, draw=mycolor2, fill=mycolor2, forget plot]
table[] {Diagramas/data/nyquist_h2-7.tsv}--cycle;
\addplot [color=mycolor2, forget plot]
  table[]{Diagramas/data/nyquist_h2-8.tsv};
\addplot[only marks, mark=*, mark options={}, mark size=1.5000pt, draw=mycolor1, fill=mycolor1] table[]{Diagramas/data/nyquist_h2-9.tsv};
\addlegendentry{Margen de ganancia}
\addplot[only marks, mark=*, mark options={}, mark size=1.5000pt, draw=mycolor4, fill=mycolor4] table[]{Diagramas/data/nyquist_h2-10.tsv};
\addlegendentry{Margen de fase}
\end{axis}
\end{tikzpicture}%

  \caption{Diagrama de Nyquist del sistema con sumador y ganancia crítica}
  \label{fig:nyquist-h2}
\end{figure}

\begin{figure}[h]
  \centering
  % This file was created by matlab2tikz.
%
%The latest updates can be retrieved from
%  http://www.mathworks.com/matlabcentral/fileexchange/22022-matlab2tikz-matlab2tikz
%where you can also make suggestions and rate matlab2tikz.
%
\begin{tikzpicture}

\begin{axis}[%
width=9.632cm,
height=2.765cm,
at={(0cm,0cm)},
scale only axis,
unbounded coords=jump,
separate axis lines,
every outer x axis line/.append style={white!15!black},
every x tick label/.append style={font=\color{white!15!black}},
every x tick/.append style={white!15!black},
xmode=log,
xmin=0.01,
xmax=15.707963267949,
xminorticks=true,
every outer y axis line/.append style={white!15!black},
every y tick label/.append style={font=\color{white!15!black}},
every y tick/.append style={white!15!black},
ymin=-720,
ymax=-90,
ytick={-720, -540, -360, -180},
ylabel style={font=\color{white!15!black}},
ylabel={Fase $[\unit{deg}]$},
axis background/.style={fill=white}
]
\addplot [color=gray, line width=1.5pt, forget plot]
  table[]{Diagramas/data/bode_h2-1.tsv};
\addplot [color=gray, line width=1.5pt, forget plot]
  table[]{Diagramas/data/bode_h2-2.tsv};
\addplot [color=gray, dashdotted, forget plot]
  table[]{Diagramas/data/bode_h2-3.tsv};
\addplot [color=gray, dashdotted, forget plot]
  table[]{Diagramas/data/bode_h2-4.tsv};
\addplot [color=mycolor2, forget plot]
  table[]{Diagramas/data/bode_h2-5.tsv};
\addplot[only marks, mark=*, mark options={}, mark size=1.5000pt, draw=mycolor2, fill=white, forget plot] table[]{Diagramas/data/bode_h2-6.tsv};
\addplot[only marks, mark=*, mark options={}, mark size=1.5000pt, draw=mycolor2, fill=mycolor2, forget plot] table[]{Diagramas/data/bode_h2-7.tsv};
\end{axis}

\begin{axis}[%
width=9.632cm,
height=2.765cm,
at={(0cm,3.235cm)},
scale only axis,
unbounded coords=jump,
separate axis lines,
every outer x axis line/.append style={white!15!black},
every x tick label/.append style={font=\color{white!15!black}},
every x tick/.append style={white!15!black},
xmode=log,
xmin=0.01,
xmax=15.707963267949,
xtick={0.01,0.1,1,10},
xticklabels={\empty},
xminorticks=true,
every outer y axis line/.append style={white!15!black},
every y tick label/.append style={font=\color{white!15!black}},
every y tick/.append style={white!15!black},
ymin=-50,
ymax=40,
ylabel style={font=\color{white!15!black}},
ylabel={Magnitud $[\unit{dB}]$},
axis background/.style={fill=white},
title style={font=\color{white!15!black}},
]
\addplot [color=gray, line width=1.5pt, forget plot]
  table[]{Diagramas/data/bode_h2-8.tsv};
\addplot [color=gray, line width=1.5pt, forget plot]
  table[]{Diagramas/data/bode_h2-9.tsv};
\addplot [color=gray, dashdotted, forget plot]
  table[]{Diagramas/data/bode_h2-10.tsv};
\addplot [color=gray, dashdotted, forget plot]
  table[]{Diagramas/data/bode_h2-11.tsv};
\addplot [color=mycolor2, forget plot]
  table[]{Diagramas/data/bode_h2-12.tsv};
\addplot[only marks, mark=*, mark options={}, mark size=1.5000pt, draw=mycolor2, fill=white, forget plot] table[]{Diagramas/data/bode_h2-13.tsv};
\addplot[only marks, mark=*, mark options={}, mark size=1.5000pt, draw=mycolor2, fill=mycolor2, forget plot] table[]{Diagramas/data/bode_h2-14.tsv};
\end{axis}
\end{tikzpicture}%

  \caption{Diagrama de Bode del sistema con sumador y ganancia crítica}
  \label{fig:bode-h2}
\end{figure}

También se graficó el sistema con el retardo crítico agregado (\autoref{fig:nyquist-h3} y
\autoref{fig:bode{h3}}). De esto, se determinó el margen de ganancia siendo así:
\begin{equation}
  \boxed{\textbf{M.G.} = 1.2664 = 0.55\ \unit{dB}}
\end{equation}

\begin{figure}[h]
  \centering
  % This file was created by matlab2tikz.
%
%The latest updates can be retrieved from
%  http://www.mathworks.com/matlabcentral/fileexchange/22022-matlab2tikz-matlab2tikz
%where you can also make suggestions and rate matlab2tikz.
%
\begin{tikzpicture}

\begin{axis}[%
width=9.785cm,
height=6cm,
at={(0cm,0cm)},
scale only axis,
unbounded coords=jump,
separate axis lines,
every outer x axis line/.append style={white!15!black},
every x tick label/.append style={font=\color{white!15!black}},
every x tick/.append style={white!15!black},
xmin=-1.2,
xmax=0.5,
every outer y axis line/.append style={white!15!black},
every y tick label/.append style={font=\color{white!15!black}},
every y tick/.append style={white!15!black},
ymin=-1.5,
ymax=1.5,
ylabel style={font=\color{white!15!black}},
ylabel={Eje imaginario},
xlabel style={font=\color{white!15!black}},
xlabel={Eje real},
axis background/.style={fill=white},
legend cell align=left,
restrict y to domain=-20:20
]
\addplot[only marks, mark=+, mark options={}, mark size=2.5000pt, draw=red, forget plot] table[]{Diagramas/data/nyquist_h3-1.tsv};
\addplot [color=gray, dotted, forget plot]
  table[]{Diagramas/data/nyquist_h3-2.tsv};
\addplot [color=gray, dotted, forget plot]
  table[]{Diagramas/data/nyquist_h3-3.tsv};
\draw[dashdotted, draw=gray] (axis cs:0,0) circle [radius=1];
\addplot [color=gray, dashdotted, forget plot]
  table[]{Diagramas/data/nyquist_h3-4.tsv};
\addplot [color=gray, dashdotted, forget plot]
  table[]{Diagramas/data/nyquist_h3-5.tsv};

\addplot[area legend, draw=mycolor2, fill=mycolor2, forget plot]
table[] {Diagramas/data/nyquist_h3-6.tsv}--cycle;

\addplot[area legend, draw=mycolor2, fill=mycolor2, forget plot]
table[] {Diagramas/data/nyquist_h3-7.tsv}--cycle;
\addplot [color=mycolor2, forget plot]
  table[]{Diagramas/data/nyquist_h3-8.tsv};
\addplot[only marks, mark=*, mark options={}, mark size=1.5000pt, draw=mycolor1, fill=mycolor1] table[]{Diagramas/data/nyquist_h3-9.tsv};
\addlegendentry{Margen de fase}
\addplot[only marks, mark=*, mark options={}, mark size=1.5000pt, draw=mycolor4, fill=mycolor4] table[]{Diagramas/data/nyquist_h3-10.tsv};
\addlegendentry{Margen de ganancia}
\end{axis}
\end{tikzpicture}%

  \caption{Diagrama de Nyquist del sistema con sumador y retardo crítico}
  \label{fig:nyquist-h3}
\end{figure}

\begin{figure}[h]
  \centering
  % This file was created by matlab2tikz.
%
%The latest updates can be retrieved from
%  http://www.mathworks.com/matlabcentral/fileexchange/22022-matlab2tikz-matlab2tikz
%where you can also make suggestions and rate matlab2tikz.
%
\begin{tikzpicture}

\begin{axis}[%
width=9.632cm,
height=2.765cm,
at={(0cm,0cm)},
scale only axis,
unbounded coords=jump,
separate axis lines,
every outer x axis line/.append style={white!15!black},
every x tick label/.append style={font=\color{white!15!black}},
every x tick/.append style={white!15!black},
xmode=log,
xmin=0.01,
xmax=15.707963267949,
xminorticks=true,
every outer y axis line/.append style={white!15!black},
every y tick label/.append style={font=\color{white!15!black}},
every y tick/.append style={white!15!black},
ymin=-900,
ymax=-90,
ytick={-900, -720, -540, -360, -180,    0},
ylabel style={font=\color{white!15!black}},
ylabel={Fase $[\unit{deg}]$},
axis background/.style={fill=white}
]
\addplot [color=gray, line width=1.5pt, forget plot]
  table[]{Diagramas/data/bode_h3-1.tsv};
\addplot [color=gray, line width=1.5pt, forget plot]
  table[]{Diagramas/data/bode_h3-2.tsv};
\addplot [color=gray, dashdotted, forget plot]
  table[]{Diagramas/data/bode_h3-3.tsv};
\addplot [color=gray, dashdotted, forget plot]
  table[]{Diagramas/data/bode_h3-4.tsv};
\addplot [color=mycolor2, forget plot]
  table[]{Diagramas/data/bode_h3-5.tsv};
\addplot[only marks, mark=*, mark options={}, mark size=1.5000pt, draw=mycolor2, fill=white, forget plot] table[]{Diagramas/data/bode_h3-6.tsv};
\addplot[only marks, mark=*, mark options={}, mark size=1.5000pt, draw=mycolor2, fill=mycolor2, forget plot] table[]{Diagramas/data/bode_h3-7.tsv};
\addplot [color=gray, dotted, forget plot]
  table[]{Diagramas/data/bode_h3-8.tsv};
\addplot [color=gray, dotted, forget plot]
  table[]{Diagramas/data/bode_h3-9.tsv};
\addplot [color=gray, line width=1.5pt, forget plot]
  table[]{Diagramas/data/bode_h3-10.tsv};
\addplot [color=gray, line width=1.5pt, forget plot]
  table[]{Diagramas/data/bode_h3-11.tsv};
\addplot [color=gray, line width=1.5pt, forget plot]
  table[]{Diagramas/data/bode_h3-12.tsv};
\addplot [color=gray, line width=1.5pt, forget plot]
  table[]{Diagramas/data/bode_h3-13.tsv};
\addplot [color=gray, dashdotted, forget plot]
  table[]{Diagramas/data/bode_h3-14.tsv};
\addplot [color=gray, dotted, forget plot]
  table[]{Diagramas/data/bode_h3-15.tsv};
\addplot [color=gray, dotted, forget plot]
  table[]{Diagramas/data/bode_h3-16.tsv};
\addplot [color=gray, line width=1.5pt, forget plot]
  table[]{Diagramas/data/bode_h3-17.tsv};
\addplot [color=gray, line width=1.5pt, forget plot]
  table[]{Diagramas/data/bode_h3-18.tsv};
\addplot [color=gray, line width=1.5pt, forget plot]
  table[]{Diagramas/data/bode_h3-19.tsv};
\addplot [color=gray, line width=1.5pt, forget plot]
  table[]{Diagramas/data/bode_h3-20.tsv};
\addplot [color=gray, dashdotted, forget plot]
  table[]{Diagramas/data/bode_h3-21.tsv};
\addplot [color=gray, dotted, forget plot]
  table[]{Diagramas/data/bode_h3-22.tsv};
\addplot [color=gray, dotted, forget plot]
  table[]{Diagramas/data/bode_h3-23.tsv};
\end{axis}

\begin{axis}[%
width=9.632cm,
height=2.765cm,
at={(0cm,3.235cm)},
scale only axis,
unbounded coords=jump,
separate axis lines,
every outer x axis line/.append style={white!15!black},
every x tick label/.append style={font=\color{white!15!black}},
every x tick/.append style={white!15!black},
xmode=log,
xmin=0.01,
xmax=15.707963267949,
xtick={0.01,0.1,1,10},
xticklabels={\empty},
xminorticks=true,
every outer y axis line/.append style={white!15!black},
every y tick label/.append style={font=\color{white!15!black}},
every y tick/.append style={white!15!black},
ymin=-60,
ymax=40,
ylabel style={font=\color{white!15!black}},
ylabel={Magnitud $[\unit{dB}]$},
axis background/.style={fill=white},
title style={font=\color{white!15!black}},
]
\addplot [color=gray, line width=1.5pt, forget plot]
  table[]{Diagramas/data/bode_h3-24.tsv};
\addplot [color=gray, line width=1.5pt, forget plot]
  table[]{Diagramas/data/bode_h3-25.tsv};
\addplot [color=gray, dashdotted, forget plot]
  table[]{Diagramas/data/bode_h3-26.tsv};
\addplot [color=gray, dashdotted, forget plot]
  table[]{Diagramas/data/bode_h3-27.tsv};
\addplot [color=mycolor2, forget plot]
  table[]{Diagramas/data/bode_h3-28.tsv};
\addplot[only marks, mark=*, mark options={}, mark size=1.5000pt, draw=mycolor2, fill=white, forget plot] table[]{Diagramas/data/bode_h3-29.tsv};
\addplot[only marks, mark=*, mark options={}, mark size=1.5000pt, draw=mycolor2, fill=mycolor2, forget plot] table[]{Diagramas/data/bode_h3-30.tsv};
\end{axis}
\end{tikzpicture}%

  \caption{Diagrama de Bode del sistema con sumador y retardo crítico}
  \label{fig:bode-h3}
\end{figure}


\FloatBarrier
\subsection{Comentarios}

\textbf{NO OLVIDAR!!!}


%%% H %%%
\begin{itemize}
  \item Idem (g) comparando con un controlador discreto con kc = 20/(z-1) incluyendo retardo.
\end{itemize}
