\section{Pregunta \texttt{f)}}\label{pregunta-f}

\subsection{Desarrollo}

Nuevamente nos piden calcular el margen de ganancia (M.G.) y margen de fase
(M.F.). solo que para un \(h_c(s)= \mora{k_c}/s = 10m/s\), Ahora bien, sabemos 
por tarea anterior \cite{tarea-2-sdc} que la F.de T. de lazo directo del sistema es:
\begin{equation}
  LD(s) = \frac{\mora{k_c}}{s} \cdot \nara{k_a} \cdot \frac{\nara{b_{0\omega}} \left(\nara{b_{1\psi}}s + \nara{b_{0\psi}}\right)}
  {\left(\nara{a_{1\omega}}s + \nara{a_{0\omega}}\right)\left(\nara{a_{2\psi}}s^2 + \nara{a_{1\psi}}s + \nara{a_{0\psi}}\right)} \cdot \nara{k_{st}}
\end{equation}

Entonces, para determinar los valores pedidos, vamos a utilizar  \textit{MATLAB}.
En específico, vamos a utilizar el comando \texttt{margin}. Este comando nos retornará
los valores de margen de fase y margen de ganancia en grados y en ganancia absoluta
respectivamente. Entonces, se tiene que estos son:

\begin{equation}
  \boxed{\textbf{M.G.} = 3.0977 = 9.8208\ \unit{dB}} \qquad \boxed{\textbf{M.F.} = \ang{68.4791}}
\end{equation}

Al igual que para el sistema anterior, se graficó el Nyquist (\autoref{fig:nyquist-f1})
y el Bode (\autoref{fig:bode-f1}) de este sistema. Además, se determinó el valor de la
ganancia crítica y del retardo crítico, obteniendo así:
\begin{equation}
  \boxed{\mora{k_{cr}} = 31.0\unit{m}} \qquad \boxed{\mora{t_{cr}} = 1.2143\ \unit{s}}
\end{equation}

\begin{figure}[h]
  \centering
  % This file was created by matlab2tikz.
%
%The latest updates can be retrieved from
%  http://www.mathworks.com/matlabcentral/fileexchange/22022-matlab2tikz-matlab2tikz
%where you can also make suggestions and rate matlab2tikz.
%
\begin{tikzpicture}

\begin{axis}[%
width=9.785cm,
height=6cm,
at={(0cm,0cm)},
scale only axis,
unbounded coords=jump,
separate axis lines,
every outer x axis line/.append style={white!15!black},
every x tick label/.append style={font=\color{white!15!black}},
every x tick/.append style={white!15!black},
xmin=-1,
xmax=0.2,
every outer y axis line/.append style={white!15!black},
every y tick label/.append style={font=\color{white!15!black}},
every y tick/.append style={white!15!black},
ymin=-2,
ymax=2,
ylabel style={font=\color{white!15!black}},
ylabel={Eje real},
xlabel={Eje imaginario},
axis background/.style={fill=white},
title style={font=\color{white!15!black}},
restrict y to domain=-20:20,
legend cell align=left,
]
\addplot[only marks, mark=+, mark options={}, mark size=2.5000pt, draw=red, forget plot] table[]{Diagramas/data/nyquist_f1-1.tsv};
\addplot [color=gray, dotted, forget plot]
  table[]{Diagramas/data/nyquist_f1-2.tsv};
\addplot [color=gray, dotted, forget plot]
  table[]{Diagramas/data/nyquist_f1-3.tsv};
\draw[dashdotted, draw=gray] (axis cs:0,0) circle [radius=1];
\addplot [color=gray, dashdotted, forget plot]
  table[]{Diagramas/data/nyquist_f1-4.tsv};
\addplot [color=gray, dashdotted, forget plot]
  table[]{Diagramas/data/nyquist_f1-5.tsv};

\addplot[area legend, draw=mycolor2, fill=mycolor2, forget plot]
table[] {Diagramas/data/nyquist_f1-6.tsv}--cycle;

\addplot[area legend, draw=mycolor2, fill=mycolor2, forget plot]
table[] {Diagramas/data/nyquist_f1-7.tsv}--cycle;
\addplot [color=mycolor2, forget plot]
  table[]{Diagramas/data/nyquist_f1-8.tsv};
\addplot[only marks, mark=*, mark options={}, mark size=1.5000pt, draw=mycolor1, fill=mycolor1] table[]{Diagramas/data/nyquist_f1-9.tsv};
  \addlegendentry{Margen de fase}
\addplot[only marks, mark=*, mark options={}, mark size=1.5000pt, draw=mycolor4, fill=mycolor4] table[]{Diagramas/data/nyquist_f1-10.tsv};
  \addlegendentry{Margen de ganancia}
\end{axis}
\end{tikzpicture}%

  \caption{Diagrama de Nyquist del sistema con $h_{c}(s) = \mora{k_{c}}\mathbin{s}$.}
  \label{fig:nyquist-f1}
\end{figure}

\begin{figure}[h]
  \centering
  % This file was created by matlab2tikz.
%
%The latest updates can be retrieved from
%  http://www.mathworks.com/matlabcentral/fileexchange/22022-matlab2tikz-matlab2tikz
%where you can also make suggestions and rate matlab2tikz.
%
\begin{tikzpicture}

\begin{axis}[%
width=9.535cm,
height=2.765cm,
at={(0cm,0cm)},
scale only axis,
unbounded coords=jump,
separate axis lines,
every outer x axis line/.append style={white!15!black},
every x tick label/.append style={font=\color{white!15!black}},
every x tick/.append style={white!15!black},
xmode=log,
xmin=0.1,
xmax=100,
xminorticks=true,
every outer y axis line/.append style={white!15!black},
every y tick label/.append style={font=\color{white!15!black}},
every y tick/.append style={white!15!black},
ymin=-280,
ymax=-90,
ytick={-315, -270, -225, -180, -135,  -90},
ylabel style={font=\color{white!15!black}},
ylabel={Phase (deg)},
axis background/.style={fill=white}
]
\addplot [color=gray, line width=1.5pt, forget plot]
  table[]{Diagramas/data/bode_f1-1.tsv};
\addplot [color=gray, line width=1.5pt, forget plot]
  table[]{Diagramas/data/bode_f1-2.tsv};
\addplot [color=gray, dashdotted, forget plot]
  table[]{Diagramas/data/bode_f1-3.tsv};
\addplot [color=gray, dashdotted, forget plot]
  table[]{Diagramas/data/bode_f1-4.tsv};
\addplot [color=mycolor2, forget plot]
  table[]{Diagramas/data/bode_f1-5.tsv};
\addplot[only marks, mark=*, mark options={}, mark size=1.5000pt, draw=mycolor2, fill=white, forget plot] table[]{Diagramas/data/bode_f1-6.tsv};
\addplot[only marks, mark=*, mark options={}, mark size=1.5000pt, draw=mycolor2, fill=mycolor2, forget plot] table[]{Diagramas/data/bode_f1-7.tsv};
\end{axis}

\begin{axis}[%
width=9.535cm,
height=2.765cm,
at={(0cm,3.235cm)},
scale only axis,
unbounded coords=jump,
separate axis lines,
every outer x axis line/.append style={white!15!black},
every x tick label/.append style={font=\color{white!15!black}},
every x tick/.append style={white!15!black},
xmode=log,
xmin=0.1,
xmax=100,
xtick={0.1,1,10,100},
xticklabels={\empty},
xminorticks=true,
every outer y axis line/.append style={white!15!black},
every y tick label/.append style={font=\color{white!15!black}},
every y tick/.append style={white!15!black},
ymin=-100,
ymax=40,
ylabel style={font=\color{white!15!black}},
ylabel={Magnitude (dB)},
axis background/.style={fill=white},
title style={font=\color{white!15!black}},
title={From: In(1)}
]
\addplot [color=gray, line width=1.5pt, forget plot]
  table[]{Diagramas/data/bode_f1-8.tsv};
\addplot [color=gray, line width=1.5pt, forget plot]
  table[]{Diagramas/data/bode_f1-9.tsv};
\addplot [color=gray, dashdotted, forget plot]
  table[]{Diagramas/data/bode_f1-10.tsv};
\addplot [color=gray, dashdotted, forget plot]
  table[]{Diagramas/data/bode_f1-11.tsv};
\addplot [color=mycolor2, forget plot]
  table[]{Diagramas/data/bode_f1-12.tsv};
\addplot[only marks, mark=*, mark options={}, mark size=1.5000pt, draw=mycolor2, fill=white, forget plot] table[]{Diagramas/data/bode_f1-13.tsv};
\addplot[only marks, mark=*, mark options={}, mark size=1.5000pt, draw=mycolor2, fill=mycolor2, forget plot] table[]{Diagramas/data/bode_f1-14.tsv};
\end{axis}
\end{tikzpicture}%

  \caption{Diagrama de Bode del sistema con $h_{c}(s) = \mora{k_{c}}\mathbin{s}$.}
  \label{fig:bode-f1}
\end{figure}

Luego, se graficó el Nyquist (\autoref{fig:nyquist-f2}) y el Bode (\autoref{fig:bode-f2})
del sistema con la ganancia crítica añadida. Para este caso, se obtuvo margen de fase
\begin{equation}
  \boxed{\textbf{M.F.} = \ang{-0.0946}}
\end{equation}

\begin{figure}[h]
  \centering
  % This file was created by matlab2tikz.
%
%The latest updates can be retrieved from
%  http://www.mathworks.com/matlabcentral/fileexchange/22022-matlab2tikz-matlab2tikz
%where you can also make suggestions and rate matlab2tikz.
%
\definecolor{mycolor1}{rgb}{0.00000,0.44700,0.74100}%
\definecolor{mycolor2}{rgb}{1.00000,0.27059,0.22745}%
\definecolor{mycolor3}{rgb}{0.49020,0.49020,0.49020}%
\definecolor{mycolor4}{rgb}{0.38039,0.38039,0.38039}%
%
\begin{tikzpicture}

\begin{axis}[%
width=9.785cm,
height=6cm,
at={(0cm,0cm)},
scale only axis,
unbounded coords=jump,
separate axis lines,
every outer x axis line/.append style={mycolor4},
every x tick label/.append style={font=\color{mycolor4}},
every x tick/.append style={mycolor4},
xmin=-1.6,
xmax=0.2,
every outer y axis line/.append style={mycolor4},
every y tick label/.append style={font=\color{mycolor4}},
every y tick/.append style={mycolor4},
ymin=-40,
ymax=40,
ylabel style={font=\color{mycolor4}},
ylabel={To: Out(1)},
axis background/.style={fill=white},
title style={font=\color{mycolor4}},
title={From: In(1)}
]
\addplot[only marks, mark=+, mark options={}, mark size=2.5000pt, draw=mycolor2, forget plot] table[]{Diagramas/data/nyquist_f2-1.tsv};
\addplot [color=mycolor3, dotted, forget plot]
  table[]{Diagramas/data/nyquist_f2-2.tsv};
\addplot [color=mycolor3, dotted, forget plot]
  table[]{Diagramas/data/nyquist_f2-3.tsv};
\draw[dashdotted, draw=mycolor3] (axis cs:-1,-1) rectangle (axis cs:1,1);
\addplot [color=mycolor3, dashdotted, forget plot]
  table[]{Diagramas/data/nyquist_f2-4.tsv};
\addplot [color=mycolor3, dashdotted, forget plot]
  table[]{Diagramas/data/nyquist_f2-5.tsv};

\addplot[area legend, draw=mycolor1, fill=mycolor1, forget plot]
table[] {Diagramas/data/nyquist_f2-6.tsv}--cycle;

\addplot[area legend, draw=mycolor1, fill=mycolor1, forget plot]
table[] {Diagramas/data/nyquist_f2-7.tsv}--cycle;
\addplot [color=mycolor1, forget plot]
  table[]{Diagramas/data/nyquist_f2-8.tsv};
\addplot[only marks, mark=*, mark options={}, mark size=1.5000pt, draw=mycolor1, fill=mycolor1, forget plot] table[]{Diagramas/data/nyquist_f2-9.tsv};
\addplot[only marks, mark=*, mark options={}, mark size=1.5000pt, draw=mycolor1, fill=mycolor1, forget plot] table[]{Diagramas/data/nyquist_f2-10.tsv};
\end{axis}

\begin{axis}[%
width=12.224cm,
height=7.334cm,
at={(-1.277cm,-0.784cm)},
scale only axis,
xmin=0,
xmax=1,
ymin=0,
ymax=1,
axis line style={draw=none},
ticks=none,
axis x line*=bottom,
axis y line*=left
]
\end{axis}
\end{tikzpicture}%
  \caption{Diagrama de Nyquist del sistema con integrador y ganancia crítica}
  \label{fig:nyquist-f2}
\end{figure}

\begin{figure}[h]
  \centering
  % This file was created by matlab2tikz.
%
%The latest updates can be retrieved from
%  http://www.mathworks.com/matlabcentral/fileexchange/22022-matlab2tikz-matlab2tikz
%where you can also make suggestions and rate matlab2tikz.
%
\begin{tikzpicture}

\begin{axis}[%
width=9.535cm,
height=2.765cm,
at={(0cm,0cm)},
scale only axis,
unbounded coords=jump,
separate axis lines,
every outer x axis line/.append style={white!15!black},
every x tick label/.append style={font=\color{white!15!black}},
every x tick/.append style={white!15!black},
xmode=log,
xmin=0.1,
xmax=100,
xminorticks=true,
every outer y axis line/.append style={white!15!black},
every y tick label/.append style={font=\color{white!15!black}},
every y tick/.append style={white!15!black},
ymin=-280,
ymax=-90,
ytick={-315, -270, -225, -180, -135,  -90},
ylabel style={font=\color{white!15!black}},
ylabel={Fase $[\unit{deg}]$},
axis background/.style={fill=white}
]
\addplot [color=gray, line width=1.5pt, forget plot]
  table[]{Diagramas/data/bode_f2-1.tsv};
\addplot [color=gray, line width=1.5pt, forget plot]
  table[]{Diagramas/data/bode_f2-2.tsv};
\addplot [color=gray, dashdotted, forget plot]
  table[]{Diagramas/data/bode_f2-3.tsv};
\addplot [color=gray, dashdotted, forget plot]
  table[]{Diagramas/data/bode_f2-4.tsv};
\addplot [color=mycolor2, forget plot]
  table[]{Diagramas/data/bode_f2-5.tsv};
\addplot[only marks, mark=*, mark options={}, mark size=1.5000pt, draw=mycolor2, fill=white, forget plot] table[]{Diagramas/data/bode_f2-6.tsv};
\addplot[only marks, mark=*, mark options={}, mark size=1.5000pt, draw=mycolor2, fill=mycolor2, forget plot] table[]{Diagramas/data/bode_f2-7.tsv};
\end{axis}

\begin{axis}[%
width=9.535cm,
height=2.765cm,
at={(0cm,3.235cm)},
scale only axis,
unbounded coords=jump,
separate axis lines,
every outer x axis line/.append style={white!15!black},
every x tick label/.append style={font=\color{white!15!black}},
every x tick/.append style={white!15!black},
xmode=log,
xmin=0.1,
xmax=100,
xtick={0.1,1,10,100},
xticklabels={\empty},
xminorticks=true,
every outer y axis line/.append style={white!15!black},
every y tick label/.append style={font=\color{white!15!black}},
every y tick/.append style={white!15!black},
ymin=-100,
ymax=40,
ylabel style={font=\color{white!15!black}},
ylabel={Magnitud $[\unit{dB}]$},
axis background/.style={fill=white},
title style={font=\color{white!15!black}},
]
\addplot [color=gray, line width=1.5pt, forget plot]
  table[]{Diagramas/data/bode_f2-8.tsv};
\addplot [color=gray, line width=1.5pt, forget plot]
  table[]{Diagramas/data/bode_f2-9.tsv};
\addplot [color=gray, dashdotted, forget plot]
  table[]{Diagramas/data/bode_f2-10.tsv};
\addplot [color=gray, dashdotted, forget plot]
  table[]{Diagramas/data/bode_f2-11.tsv};
\addplot [color=mycolor2, forget plot]
  table[]{Diagramas/data/bode_f2-12.tsv};
\addplot[only marks, mark=*, mark options={}, mark size=1.5000pt, draw=mycolor2, fill=white, forget plot] table[]{Diagramas/data/bode_f2-13.tsv};
\addplot[only marks, mark=*, mark options={}, mark size=1.5000pt, draw=mycolor2, fill=mycolor2, forget plot] table[]{Diagramas/data/bode_f2-14.tsv};
\end{axis}
\end{tikzpicture}%

  \caption{Diagrama de Bode del sistema con integrador y ganancia crítica}
  \label{fig:bode-f2}
\end{figure}

Y finalmente, se hizo lo mismo que antes, pero con el retardo crítico (\autoref{fig:nyquist-f3}
y \autoref{fig:bode-f3}), donde se obtuvo un margen de ganancia
\begin{equation}
  \boxed{\textbf{M.G.} = 1 = 0\ \unit{dB}}
\end{equation}

\begin{figure}[h]
  \centering
  % This file was created by matlab2tikz.
%
%The latest updates can be retrieved from
%  http://www.mathworks.com/matlabcentral/fileexchange/22022-matlab2tikz-matlab2tikz
%where you can also make suggestions and rate matlab2tikz.
%
\begin{tikzpicture}

\begin{axis}[%
width=9.785cm,
height=6cm,
at={(0cm,0cm)},
scale only axis,
unbounded coords=jump,
separate axis lines,
every outer x axis line/.append style={white!15!black},
every x tick label/.append style={font=\color{white!15!black}},
every x tick/.append style={white!15!black},
xmin=-2,
xmax=0.5,
every outer y axis line/.append style={white!15!black},
every y tick label/.append style={font=\color{white!15!black}},
every y tick/.append style={white!15!black},
ymin=-5,
ymax=5,
ylabel style={font=\color{white!15!black}},
ylabel={Eje imaginario},
xlabel style={font=\color{white!15!black}},
xlabel={Eje real},
axis background/.style={fill=white},
legend cell align=left,
restrict y to domain=-20:20
]
\addplot[only marks, mark=+, mark options={}, mark size=2.5000pt, draw=red, forget plot] table[]{Diagramas/data/nyquist_f3-1.tsv};
\addplot [color=gray, dotted, forget plot]
  table[]{Diagramas/data/nyquist_f3-2.tsv};
\addplot [color=gray, dotted, forget plot]
  table[]{Diagramas/data/nyquist_f3-3.tsv};
\draw[dashdotted, draw=gray] (axis cs:0,0) circle [radius=1];
\addplot [color=gray, dashdotted, forget plot]
  table[]{Diagramas/data/nyquist_f3-4.tsv};
\addplot [color=gray, dashdotted, forget plot]
  table[]{Diagramas/data/nyquist_f3-5.tsv};

\addplot[area legend, draw=mycolor2, fill=mycolor2, forget plot]
table[] {Diagramas/data/nyquist_f3-6.tsv}--cycle;

\addplot[area legend, draw=mycolor2, fill=mycolor2, forget plot]
table[] {Diagramas/data/nyquist_f3-7.tsv}--cycle;
\addplot [color=mycolor2, forget plot]
  table[]{Diagramas/data/nyquist_f3-8.tsv};
\addplot[only marks, mark=*, mark options={}, mark size=1.5000pt, draw=mycolor1, fill=mycolor1] table[]{Diagramas/data/nyquist_f3-9.tsv};
  \addlegendentry{Margen de ganancia}
\addplot[only marks, mark=*, mark options={}, mark size=1.5000pt, draw=mycolor4, fill=mycolor4] table[]{Diagramas/data/nyquist_f3-10.tsv};
  \addlegendentry{Margen de fase}
\end{axis}
\end{tikzpicture}%

  \caption{Diagrama de Nyquist del sistema con integrador y retardo crítico}
  \label{fig:nyquist-f3}
\end{figure}

\begin{figure}[h]
  \centering
  % This file was created by matlab2tikz.
%
%The latest updates can be retrieved from
%  http://www.mathworks.com/matlabcentral/fileexchange/22022-matlab2tikz-matlab2tikz
%where you can also make suggestions and rate matlab2tikz.
%
\definecolor{mycolor1}{rgb}{0.49020,0.49020,0.49020}%
\definecolor{mycolor2}{rgb}{0.00000,0.44700,0.74100}%
\definecolor{mycolor3}{rgb}{0.38039,0.38039,0.38039}%
\definecolor{mycolor4}{rgb}{0.12941,0.12941,0.12941}%
%
\begin{tikzpicture}

\begin{axis}[%
width=9.467cm,
height=2.765cm,
at={(0cm,0cm)},
scale only axis,
unbounded coords=jump,
separate axis lines,
every outer x axis line/.append style={mycolor3},
every x tick label/.append style={font=\color{mycolor3}},
every x tick/.append style={mycolor3},
xmode=log,
xmin=0.1,
xmax=100,
xminorticks=true,
every outer y axis line/.append style={mycolor3},
every y tick label/.append style={font=\color{mycolor3}},
every y tick/.append style={mycolor3},
ymin=-7200,
ymax=-0,
ytick={-7200, -5760, -4320, -2880, -1440,     0},
ylabel style={font=\color{mycolor4}},
ylabel={Phase (deg)},
axis background/.style={fill=white}
]
\addplot [color=mycolor1, line width=1.5pt, forget plot]
  table[]{Diagramas/data/bode_f3-1.tsv};
\addplot [color=mycolor1, line width=1.5pt, forget plot]
  table[]{Diagramas/data/bode_f3-2.tsv};
\addplot [color=mycolor1, dashdotted, forget plot]
  table[]{Diagramas/data/bode_f3-3.tsv};
\addplot [color=mycolor1, dashdotted, forget plot]
  table[]{Diagramas/data/bode_f3-4.tsv};
\addplot [color=mycolor2, forget plot]
  table[]{Diagramas/data/bode_f3-5.tsv};
\addplot[only marks, mark=*, mark options={}, mark size=1.5000pt, draw=mycolor2, fill=white, forget plot] table[]{Diagramas/data/bode_f3-6.tsv};
\addplot[only marks, mark=*, mark options={}, mark size=1.5000pt, draw=mycolor2, fill=mycolor2, forget plot] table[]{Diagramas/data/bode_f3-7.tsv};
\addplot [color=mycolor1, dotted, forget plot]
  table[]{Diagramas/data/bode_f3-8.tsv};
\addplot [color=mycolor1, dotted, forget plot]
  table[]{Diagramas/data/bode_f3-9.tsv};
\addplot [color=mycolor1, line width=1.5pt, forget plot]
  table[]{Diagramas/data/bode_f3-10.tsv};
\addplot [color=mycolor1, line width=1.5pt, forget plot]
  table[]{Diagramas/data/bode_f3-11.tsv};
\addplot [color=mycolor1, line width=1.5pt, forget plot]
  table[]{Diagramas/data/bode_f3-12.tsv};
\addplot [color=mycolor1, line width=1.5pt, forget plot]
  table[]{Diagramas/data/bode_f3-13.tsv};
\addplot [color=mycolor1, dashdotted, forget plot]
  table[]{Diagramas/data/bode_f3-14.tsv};
\addplot [color=mycolor1, dotted, forget plot]
  table[]{Diagramas/data/bode_f3-15.tsv};
\addplot [color=mycolor1, dotted, forget plot]
  table[]{Diagramas/data/bode_f3-16.tsv};
\addplot [color=mycolor1, line width=1.5pt, forget plot]
  table[]{Diagramas/data/bode_f3-17.tsv};
\addplot [color=mycolor1, line width=1.5pt, forget plot]
  table[]{Diagramas/data/bode_f3-18.tsv};
\addplot [color=mycolor1, line width=1.5pt, forget plot]
  table[]{Diagramas/data/bode_f3-19.tsv};
\addplot [color=mycolor1, line width=1.5pt, forget plot]
  table[]{Diagramas/data/bode_f3-20.tsv};
\addplot [color=mycolor1, dashdotted, forget plot]
  table[]{Diagramas/data/bode_f3-21.tsv};
\addplot [color=mycolor1, dotted, forget plot]
  table[]{Diagramas/data/bode_f3-22.tsv};
\addplot [color=mycolor1, dotted, forget plot]
  table[]{Diagramas/data/bode_f3-23.tsv};
\end{axis}

\begin{axis}[%
width=9.467cm,
height=2.765cm,
at={(0cm,3.235cm)},
scale only axis,
unbounded coords=jump,
separate axis lines,
every outer x axis line/.append style={mycolor3},
every x tick label/.append style={font=\color{mycolor3}},
every x tick/.append style={mycolor3},
xmode=log,
xmin=0.1,
xmax=100,
xtick={0.1,1,10,100},
xticklabels={\empty},
xminorticks=true,
every outer y axis line/.append style={mycolor3},
every y tick label/.append style={font=\color{mycolor3}},
every y tick/.append style={mycolor3},
ymin=-100,
ymax=40,
ylabel style={font=\color{mycolor4}},
ylabel={Magnitude (dB)},
axis background/.style={fill=white},
title style={font=\color{mycolor3}},
title={From: In(1)}
]
\addplot [color=mycolor1, line width=1.5pt, forget plot]
  table[]{Diagramas/data/bode_f3-24.tsv};
\addplot [color=mycolor1, line width=1.5pt, forget plot]
  table[]{Diagramas/data/bode_f3-25.tsv};
\addplot [color=mycolor1, dashdotted, forget plot]
  table[]{Diagramas/data/bode_f3-26.tsv};
\addplot [color=mycolor1, dashdotted, forget plot]
  table[]{Diagramas/data/bode_f3-27.tsv};
\addplot [color=mycolor2, forget plot]
  table[]{Diagramas/data/bode_f3-28.tsv};
\addplot[only marks, mark=*, mark options={}, mark size=1.5000pt, draw=mycolor2, fill=white, forget plot] table[]{Diagramas/data/bode_f3-29.tsv};
\addplot[only marks, mark=*, mark options={}, mark size=1.5000pt, draw=mycolor2, fill=mycolor2, forget plot] table[]{Diagramas/data/bode_f3-30.tsv};
\end{axis}

\begin{axis}[%
width=10.298cm,
height=6.179cm,
at={(-0.729cm,-0.135cm)},
scale only axis,
xmin=0,
xmax=1,
ymin=0,
ymax=1,
axis line style={draw=none},
ticks=none,
axis x line*=bottom,
axis y line*=left
]
\end{axis}
\end{tikzpicture}%
  \caption{Diagrama de Bode del sistema con integrador y retardo crítico}
  \label{fig:bode-f3}
\end{figure}

\FloatBarrier
\subsection{Comentarios}

Se puede observar entonces que para este sistema, con un controlador tipo integrador,
nuevamente se tiene que la ganancia crítica es apróximadamente la misma a la que se
obtuvo en la tarea 2 \cite{tarea-2-sdc}. Además, se observa que se tiene un mayor
margen de fase en relación al sistema con controlador proporcional, por lo que, si
se implementara el sistema, sería más difícil que capacitancias parásitas por ejemplo
hicieran del sistema inestable.

Del Nyquist, se puede ver que este tiende hacia el infinito, a diferencia del Nyquist
anterior, el cual se mantenía acotado a una región del plano.

