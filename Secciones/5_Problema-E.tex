\section{Pregunta \texttt{e)}}\label{pregunta-e}

\subsection{Desarrollo}

Al igual que antes, se vuelve a graficar Nyquist (\autoref{fig:nyquist-e}) y
Bode (\autoref{fig:bode-e}), pero se le agregó el retardo crítico determinado
en la pregunta \hyperref[pregunta-c]{\texttt{c)}}.

Entonces, la ganancia crítica para este caso es
\begin{equation}
  \boxed{\textbf{M.G.} = 0.9994 = -0.0120\ \unit{dB}}
\end{equation}

\begin{figure}[h]
  \centering
  % This file was created by matlab2tikz.
%
%The latest updates can be retrieved from
%  http://www.mathworks.com/matlabcentral/fileexchange/22022-matlab2tikz-matlab2tikz
%where you can also make suggestions and rate matlab2tikz.
%
\begin{tikzpicture}

\begin{axis}[%
width=9.785cm,
height=6cm,
at={(0cm,0cm)},
scale only axis,
unbounded coords=jump,
separate axis lines,
every outer x axis line/.append style={white!15!black},
every x tick label/.append style={font=\color{white!15!black}},
every x tick/.append style={white!15!black},
xmin=-2,
xmax=1.5,
every outer y axis line/.append style={white!15!black},
every y tick label/.append style={font=\color{white!15!black}},
every y tick/.append style={white!15!black},
ymin=-3,
ymax=3,
ylabel style={font=\color{white!15!black}},
ylabel={Eje imaginario},
xlabel style={font=\color{white!15!black}},
xlabel={Eje real},
axis background/.style={fill=white},
legend cell align=left,
]
\addplot[only marks, mark=+, mark options={}, mark size=2.5000pt, draw=red, forget plot] table[]{Diagramas/data/nyquist_e-1.tsv};
\addplot [color=gray, dotted, forget plot]
  table[]{Diagramas/data/nyquist_e-2.tsv};
\addplot [color=gray, dotted, forget plot]
  table[]{Diagramas/data/nyquist_e-3.tsv};
\draw[dashdotted, draw=gray] (axis cs:0,0) circle [radius=1];
\addplot [color=gray, dashdotted, forget plot]
  table[]{Diagramas/data/nyquist_e-4.tsv};
\addplot [color=gray, dashdotted, forget plot]
  table[]{Diagramas/data/nyquist_e-5.tsv};

\addplot[area legend, draw=mycolor2, fill=mycolor2, forget plot]
table[] {Diagramas/data/nyquist_e-6.tsv}--cycle;

\addplot[area legend, draw=mycolor2, fill=mycolor2, forget plot]
table[] {Diagramas/data/nyquist_e-7.tsv}--cycle;
\addplot [color=mycolor2, forget plot]
  table[]{Diagramas/data/nyquist_e-8.tsv};
\addplot[only marks, mark=*, mark options={}, mark size=1.5000pt, draw=mycolor1, fill=mycolor1] table[]{Diagramas/data/nyquist_e-9.tsv};
\addlegendentry{Margen de ganancia}
\addplot[only marks, mark=*, mark options={}, mark size=1.5000pt, draw=mycolor4, fill=mycolor4] table[]{Diagramas/data/nyquist_e-10.tsv};
\addlegendentry{Margen de fase}
\end{axis}
\end{tikzpicture}%

  \caption{Diagrama de Nyquist del sistema con retardo crítico.}
  \label{fig:nyquist-e}
\end{figure}

\begin{figure}[h]
  \centering
  % This file was created by matlab2tikz.
%
%The latest updates can be retrieved from
%  http://www.mathworks.com/matlabcentral/fileexchange/22022-matlab2tikz-matlab2tikz
%where you can also make suggestions and rate matlab2tikz.
%
\begin{tikzpicture}

\begin{axis}[%
width=9.535cm,
height=2.765cm,
at={(0cm,0cm)},
scale only axis,
unbounded coords=jump,
separate axis lines,
every outer x axis line/.append style={white!15!black},
every x tick label/.append style={font=\color{white!15!black}},
every x tick/.append style={white!15!black},
xmode=log,
xmin=0.1,
xmax=100,
xminorticks=true,
every outer y axis line/.append style={white!15!black},
every y tick label/.append style={font=\color{white!15!black}},
every y tick/.append style={white!15!black},
ymin=-315,
ymax=-0,
ytick={-270, -180, -90, 0},
ylabel style={font=\color{white!15!black}},
ylabel={Fase $[\unit{deg}]$},
axis background/.style={fill=white}
]
\addplot [color=gray, line width=1.5pt, forget plot]
  table[]{Diagramas/data/bode_e-1.tsv};
\addplot [color=gray, line width=1.5pt, forget plot]
  table[]{Diagramas/data/bode_e-2.tsv};
\addplot [color=gray, dashdotted, forget plot]
  table[]{Diagramas/data/bode_e-3.tsv};
\addplot [color=gray, dashdotted, forget plot]
  table[]{Diagramas/data/bode_e-4.tsv};
\addplot [color=mycolor2, forget plot]
  table[]{Diagramas/data/bode_e-5.tsv};
\addplot[only marks, mark=*, mark options={}, mark size=1.5000pt, draw=mycolor2, fill=white, forget plot] table[]{Diagramas/data/bode_e-6.tsv};
\addplot[only marks, mark=*, mark options={}, mark size=1.5000pt, draw=mycolor2, fill=mycolor2, forget plot] table[]{Diagramas/data/bode_e-7.tsv};
\addplot [color=gray, dotted, forget plot]
  table[]{Diagramas/data/bode_e-8.tsv};
\addplot [color=gray, dotted, forget plot]
  table[]{Diagramas/data/bode_e-9.tsv};
\end{axis}

\begin{axis}[%
width=9.535cm,
height=2.765cm,
at={(0cm,3.235cm)},
scale only axis,
unbounded coords=jump,
separate axis lines,
every outer x axis line/.append style={white!15!black},
every x tick label/.append style={font=\color{white!15!black}},
every x tick/.append style={white!15!black},
xmode=log,
xmin=0.1,
xmax=100,
xtick={0.1,1,10,100},
xticklabels={\empty},
xminorticks=true,
every outer y axis line/.append style={white!15!black},
every y tick label/.append style={font=\color{white!15!black}},
every y tick/.append style={white!15!black},
ymin=-60,
ymax=10,
ylabel style={font=\color{white!15!black}},
ylabel={Magnitud $[\unit{dB}]$},
axis background/.style={fill=white},
title style={font=\color{white!15!black}},
]
\addplot [color=gray, line width=1.5pt, forget plot]
  table[]{Diagramas/data/bode_e-10.tsv};
\addplot [color=gray, line width=1.5pt, forget plot]
  table[]{Diagramas/data/bode_e-11.tsv};
\addplot [color=gray, dashdotted, forget plot]
  table[]{Diagramas/data/bode_e-12.tsv};
\addplot [color=gray, dashdotted, forget plot]
  table[]{Diagramas/data/bode_e-13.tsv};
\addplot [color=mycolor2, forget plot]
  table[]{Diagramas/data/bode_e-14.tsv};
\addplot[only marks, mark=*, mark options={}, mark size=1.5000pt, draw=mycolor2, fill=white, forget plot] table[]{Diagramas/data/bode_e-15.tsv};
\addplot[only marks, mark=*, mark options={}, mark size=1.5000pt, draw=mycolor2, fill=mycolor2, forget plot] table[]{Diagramas/data/bode_e-16.tsv};
\end{axis}
\end{tikzpicture}%

  \caption{Diagrama de Bode del sistema con retardo crítico.}
  \label{fig:bode-e}
\end{figure}


\FloatBarrier
\subsection{Comentarios}


Ahora realizando el diagrama de Nyquist con el tiempo de retardo critico podemos ver que este no se expande, eh inicia en el mismo punto que el visto en \hyperref[pregunta-c]{\texttt{b)}} pero la diferencia es que ahora llega al punto critico de \((-1,0)\), donde con este tiempo de retardo podemos decir que el sistema se hace marginalmente estable. 

Ahora bien por lo anterior vemos que nuestro M.G es igual a uno, es decir no tenemos mas margen de ganancia para el \(\mora{k_c}\) ya que si lo aumentamos un poco, el sistema se hace inestable, ya que como vimos en \hyperref[pregunta-c]{\texttt{d)}} si aumentábamos el \(\mora{k_c}\) se expandía el diagrama de Nyquist,.



