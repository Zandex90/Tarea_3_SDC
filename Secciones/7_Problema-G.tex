\section{Pregunta \texttt{g)}}\label{pregunta-g}


\subsection{Desarrollo}

\subsubsection{Parte a}
Nuevamente nos piden calcular el margen de ganancia (M.G.) y margen de fase
(M.F.). solo que para un \(h_c(z)= \mora{k_c} = 4m\), Ahora bien, sabemos 
que la F.de T. de lazo directo del sistema discreto es:
\begin{equation}
  LD(z) = \mora{k_{c}} \nara{ k_{a}} z^{-1} \cdot \frac{\rojo{\psi}(z)}{\verd{v_i}(z)} \nara{k_{st}}  
\end{equation}

Nuevamente para determinar los valores pedidos, vamos a utilizar  \textit{MATLAB}. En específico, vamos a utilizar el comando \texttt{margin}. Este comando nos retornará los valores de margen de fase y margen de ganancia en grados y en ganancia absoluta respectivamente. Entonces, se tiene que estos son:

\begin{equation}
  \boxed{\textbf{M.G.} = 1.7174 = 4.6975\ \unit{dB}} \qquad \boxed{\textbf{M.F.} = \ang{-67.6370}}
\end{equation}


\subsubsection{Parte b}

\begin{figure}[h]
  \centering
  % This file was created by matlab2tikz.
%
%The latest updates can be retrieved from
%  http://www.mathworks.com/matlabcentral/fileexchange/22022-matlab2tikz-matlab2tikz
%where you can also make suggestions and rate matlab2tikz.
%
\definecolor{mycolor1}{rgb}{0.00000,0.44700,0.74100}%
\definecolor{mycolor2}{rgb}{1.00000,0.27059,0.22745}%
\definecolor{mycolor3}{rgb}{0.49020,0.49020,0.49020}%
\definecolor{mycolor4}{rgb}{0.38039,0.38039,0.38039}%
%
\begin{tikzpicture}

\begin{axis}[%
width=9.785cm,
height=6cm,
at={(0cm,0cm)},
scale only axis,
unbounded coords=jump,
separate axis lines,
every outer x axis line/.append style={mycolor4},
every x tick label/.append style={font=\color{mycolor4}},
every x tick/.append style={mycolor4},
xmin=-1,
xmax=0.6,
every outer y axis line/.append style={mycolor4},
every y tick label/.append style={font=\color{mycolor4}},
every y tick/.append style={mycolor4},
ymin=-0.8,
ymax=0.8,
ylabel style={font=\color{mycolor4}},
ylabel={To: Out(1)},
axis background/.style={fill=white},
title style={font=\color{mycolor4}},
title={From: In(1)}
]
\addplot[only marks, mark=+, mark options={}, mark size=2.5000pt, draw=mycolor2, forget plot] table[]{Diagramas/data/nyquist_g1-1.tsv};
\addplot [color=mycolor3, dotted, forget plot]
  table[]{Diagramas/data/nyquist_g1-2.tsv};
\addplot [color=mycolor3, dotted, forget plot]
  table[]{Diagramas/data/nyquist_g1-3.tsv};
\draw[dashdotted, draw=mycolor3] (axis cs:0,0) circle [radius=1];
\addplot [color=mycolor3, dashdotted, forget plot]
  table[]{Diagramas/data/nyquist_g1-4.tsv};
\addplot [color=mycolor3, dashdotted, forget plot]
  table[]{Diagramas/data/nyquist_g1-5.tsv};

\addplot[area legend, draw=mycolor1, fill=mycolor1, forget plot]
table[] {Diagramas/data/nyquist_g1-6.tsv}--cycle;

\addplot[area legend, draw=mycolor1, fill=mycolor1, forget plot]
table[] {Diagramas/data/nyquist_g1-7.tsv}--cycle;
\addplot [color=mycolor1, forget plot]
  table[]{Diagramas/data/nyquist_g1-8.tsv};
\addplot[only marks, mark=*, mark options={}, mark size=1.5000pt, draw=mycolor1, fill=mycolor1, forget plot] table[]{Diagramas/data/nyquist_g1-9.tsv};
\addplot[only marks, mark=*, mark options={}, mark size=1.5000pt, draw=mycolor1, fill=mycolor1, forget plot] table[]{Diagramas/data/nyquist_g1-10.tsv};
\end{axis}
\end{tikzpicture}%

  \caption{Diagrama de Nyquist del sistema con $h_{c}(z) = \mora{k_{c}}$.}
  \label{fig:nyquist-g1}
\end{figure}

\begin{figure}[h]
  \centering
  % This file was created by matlab2tikz.
%
%The latest updates can be retrieved from
%  http://www.mathworks.com/matlabcentral/fileexchange/22022-matlab2tikz-matlab2tikz
%where you can also make suggestions and rate matlab2tikz.
%
\definecolor{mycolor1}{rgb}{0.49020,0.49020,0.49020}%
\definecolor{mycolor2}{rgb}{0.00000,0.44700,0.74100}%
\definecolor{mycolor3}{rgb}{0.38039,0.38039,0.38039}%
\definecolor{mycolor4}{rgb}{0.12941,0.12941,0.12941}%
%
\begin{tikzpicture}

\begin{axis}[%
width=9.632cm,
height=2.765cm,
at={(0cm,0cm)},
scale only axis,
unbounded coords=jump,
separate axis lines,
every outer x axis line/.append style={mycolor3},
every x tick label/.append style={font=\color{mycolor3}},
every x tick/.append style={mycolor3},
xmode=log,
xmin=0.1,
xmax=15.707963267949,
xminorticks=true,
every outer y axis line/.append style={mycolor3},
every y tick label/.append style={font=\color{mycolor3}},
every y tick/.append style={mycolor3},
ymin=-360,
ymax=-0,
ytick={-360, -270, -180,  -90,    0},
ylabel style={font=\color{mycolor4}},
ylabel={Phase (deg)},
axis background/.style={fill=white}
]
\addplot [color=mycolor1, line width=1.5pt, forget plot]
  table[]{Diagramas/data/bode_g1-1.tsv};
\addplot [color=mycolor1, line width=1.5pt, forget plot]
  table[]{Diagramas/data/bode_g1-2.tsv};
\addplot [color=mycolor1, dashdotted, forget plot]
  table[]{Diagramas/data/bode_g1-3.tsv};
\addplot [color=mycolor1, dashdotted, forget plot]
  table[]{Diagramas/data/bode_g1-4.tsv};
\addplot [color=mycolor2, forget plot]
  table[]{Diagramas/data/bode_g1-5.tsv};
\addplot[only marks, mark=*, mark options={}, mark size=1.5000pt, draw=mycolor2, fill=white, forget plot] table[]{Diagramas/data/bode_g1-6.tsv};
\addplot[only marks, mark=*, mark options={}, mark size=1.5000pt, draw=mycolor2, fill=mycolor2, forget plot] table[]{Diagramas/data/bode_g1-7.tsv};
\end{axis}

\begin{axis}[%
width=9.632cm,
height=2.765cm,
at={(0cm,3.235cm)},
scale only axis,
unbounded coords=jump,
separate axis lines,
every outer x axis line/.append style={mycolor3},
every x tick label/.append style={font=\color{mycolor3}},
every x tick/.append style={mycolor3},
xmode=log,
xmin=0.1,
xmax=15.707963267949,
xtick={0.1,1,10},
xticklabels={\empty},
xminorticks=true,
every outer y axis line/.append style={mycolor3},
every y tick label/.append style={font=\color{mycolor3}},
every y tick/.append style={mycolor3},
ymin=-50,
ymax=-0,
ylabel style={font=\color{mycolor4}},
ylabel={Magnitude (dB)},
axis background/.style={fill=white},
title style={font=\color{mycolor3}},
title={From: In(1)}
]
\addplot [color=mycolor1, line width=1.5pt, forget plot]
  table[]{Diagramas/data/bode_g1-8.tsv};
\addplot [color=mycolor1, line width=1.5pt, forget plot]
  table[]{Diagramas/data/bode_g1-9.tsv};
\addplot [color=mycolor1, dashdotted, forget plot]
  table[]{Diagramas/data/bode_g1-10.tsv};
\addplot [color=mycolor1, dashdotted, forget plot]
  table[]{Diagramas/data/bode_g1-11.tsv};
\addplot [color=mycolor2, forget plot]
  table[]{Diagramas/data/bode_g1-12.tsv};
\addplot[only marks, mark=*, mark options={}, mark size=1.5000pt, draw=mycolor2, fill=white, forget plot] table[]{Diagramas/data/bode_g1-13.tsv};
\addplot[only marks, mark=*, mark options={}, mark size=1.5000pt, draw=mycolor2, fill=mycolor2, forget plot] table[]{Diagramas/data/bode_g1-14.tsv};
\end{axis}
\end{tikzpicture}%

  \caption{Diagrama de Bode del sistema con $h_{c}(z) = \mora{k_{c}}$.}
  \label{fig:bode-g1}
\end{figure}

\subsubsection{Parte d}

\begin{figure}[h]
  \centering
  % This file was created by matlab2tikz.
%
%The latest updates can be retrieved from
%  http://www.mathworks.com/matlabcentral/fileexchange/22022-matlab2tikz-matlab2tikz
%where you can also make suggestions and rate matlab2tikz.
%
\begin{tikzpicture}

\begin{axis}[%
width=9.785cm,
height=6cm,
at={(0cm,0cm)},
scale only axis,
unbounded coords=jump,
separate axis lines,
every outer x axis line/.append style={white!15!black},
every x tick label/.append style={font=\color{white!15!black}},
every x tick/.append style={white!15!black},
xmin=-1.2,
xmax=0.6,
every outer y axis line/.append style={white!15!black},
every y tick label/.append style={font=\color{white!15!black}},
every y tick/.append style={white!15!black},
ymin=-1,
ymax=1,
ylabel style={font=\color{white!15!black}},
ylabel={Eje imaginario},
xlabel style={font=\color{white!15!black}},
xlabel={Eje real},
axis background/.style={fill=white},
legend cell align=left,
restrict y to domain=-20:20
]
\addplot[only marks, mark=+, mark options={}, mark size=2.5000pt, draw=red, forget plot] table[]{Diagramas/data/nyquist_g2-1.tsv};
\addplot [color=gray, dotted, forget plot]
  table[]{Diagramas/data/nyquist_g2-2.tsv};
\addplot [color=gray, dotted, forget plot]
  table[]{Diagramas/data/nyquist_g2-3.tsv};
\draw[dashdotted, draw=gray] (axis cs:0,0) circle [radius=1];
\addplot [color=gray, dashdotted, forget plot]
  table[]{Diagramas/data/nyquist_g2-4.tsv};
\addplot [color=gray, dashdotted, forget plot]
  table[]{Diagramas/data/nyquist_g2-5.tsv};

\addplot[area legend, draw=mycolor2, fill=mycolor2, forget plot]
table[] {Diagramas/data/nyquist_g2-6.tsv}--cycle;

\addplot[area legend, draw=mycolor2, fill=mycolor2, forget plot]
table[] {Diagramas/data/nyquist_g2-7.tsv}--cycle;
\addplot [color=mycolor2, forget plot]
  table[]{Diagramas/data/nyquist_g2-8.tsv};
\addplot[only marks, mark=*, mark options={}, mark size=1.5000pt, draw=mycolor1, fill=mycolor1] table[]{Diagramas/data/nyquist_g2-10.tsv};
  \addlegendentry{Margen de ganancia}
\addplot[only marks, mark=*, mark options={}, mark size=1.5000pt, draw=mycolor4, fill=mycolor4] table[]{Diagramas/data/nyquist_g2-9.tsv};
  \addlegendentry{Margen de fase}
\end{axis}
\end{tikzpicture}%

  \caption{Diagrama de Nyquist del sistema con ganancia crítica}
  \label{fig:nyquist-g2}
\end{figure}

\begin{figure}[h]
  \centering
  % This file was created by matlab2tikz.
%
%The latest updates can be retrieved from
%  http://www.mathworks.com/matlabcentral/fileexchange/22022-matlab2tikz-matlab2tikz
%where you can also make suggestions and rate matlab2tikz.
%
\begin{tikzpicture}

\begin{axis}[%
width=9.632cm,
height=2.765cm,
at={(0cm,0cm)},
scale only axis,
unbounded coords=jump,
separate axis lines,
every outer x axis line/.append style={white!15!black},
every x tick label/.append style={font=\color{white!15!black}},
every x tick/.append style={white!15!black},
xmode=log,
xmin=0.1,
xmax=15.707963267949,
xminorticks=true,
every outer y axis line/.append style={white!15!black},
every y tick label/.append style={font=\color{white!15!black}},
every y tick/.append style={white!15!black},
ymin=-540,
ymax=-0,
ytick={-540, -360, -180,    0},
ylabel style={font=\color{white!15!black}},
ylabel={Fase $[\unit{deg}]$},
axis background/.style={fill=white}
]
\addplot [color=gray, line width=1.5pt, forget plot]
  table[]{Diagramas/data/bode_g2-1.tsv};
\addplot [color=gray, line width=1.5pt, forget plot]
  table[]{Diagramas/data/bode_g2-2.tsv};
\addplot [color=gray, dashdotted, forget plot]
  table[]{Diagramas/data/bode_g2-3.tsv};
\addplot [color=gray, dashdotted, forget plot]
  table[]{Diagramas/data/bode_g2-4.tsv};
\addplot [color=mycolor2, forget plot]
  table[]{Diagramas/data/bode_g2-5.tsv};
\addplot[only marks, mark=*, mark options={}, mark size=1.5000pt, draw=mycolor2, fill=white, forget plot] table[]{Diagramas/data/bode_g2-6.tsv};
\addplot[only marks, mark=*, mark options={}, mark size=1.5000pt, draw=mycolor2, fill=mycolor2, forget plot] table[]{Diagramas/data/bode_g2-7.tsv};
\end{axis}

\begin{axis}[%
width=9.632cm,
height=2.765cm,
at={(0cm,3.235cm)},
scale only axis,
unbounded coords=jump,
separate axis lines,
every outer x axis line/.append style={white!15!black},
every x tick label/.append style={font=\color{white!15!black}},
every x tick/.append style={white!15!black},
xmode=log,
xmin=0.1,
xmax=15.707963267949,
xtick={0.1,1,10},
xticklabels={\empty},
xminorticks=true,
every outer y axis line/.append style={white!15!black},
every y tick label/.append style={font=\color{white!15!black}},
every y tick/.append style={white!15!black},
ymin=-45,
ymax=5,
ylabel style={font=\color{white!15!black}},
ylabel={Magnitud $[\unit{dB}]$},
axis background/.style={fill=white},
title style={font=\color{white!15!black}},
]
\addplot [color=gray, line width=1.5pt, forget plot]
  table[]{Diagramas/data/bode_g2-8.tsv};
\addplot [color=gray, line width=1.5pt, forget plot]
  table[]{Diagramas/data/bode_g2-9.tsv};
\addplot [color=gray, dashdotted, forget plot]
  table[]{Diagramas/data/bode_g2-10.tsv};
\addplot [color=gray, dashdotted, forget plot]
  table[]{Diagramas/data/bode_g2-11.tsv};
\addplot [color=mycolor2, forget plot]
  table[]{Diagramas/data/bode_g2-12.tsv};
\addplot[only marks, mark=*, mark options={}, mark size=1.5000pt, draw=mycolor2, fill=white, forget plot] table[]{Diagramas/data/bode_g2-13.tsv};
\addplot[only marks, mark=*, mark options={}, mark size=1.5000pt, draw=mycolor2, fill=mycolor2, forget plot] table[]{Diagramas/data/bode_g2-14.tsv};
\end{axis}
\end{tikzpicture}%

  \caption{Diagrama de Bode del sistema con ganancia crítica}
  \label{fig:bode-g2}
\end{figure}

\subsubsection{Parte e}

\begin{figure}[h]
  \centering
  % This file was created by matlab2tikz.
%
%The latest updates can be retrieved from
%  http://www.mathworks.com/matlabcentral/fileexchange/22022-matlab2tikz-matlab2tikz
%where you can also make suggestions and rate matlab2tikz.
%
\begin{tikzpicture}

\begin{axis}[%
width=9.785cm,
height=6cm,
at={(0cm,0cm)},
scale only axis,
unbounded coords=jump,
separate axis lines,
every outer x axis line/.append style={white!15!black},
every x tick label/.append style={font=\color{white!15!black}},
every x tick/.append style={white!15!black},
xmin=-1,
xmax=0.6,
every outer y axis line/.append style={white!15!black},
every y tick label/.append style={font=\color{white!15!black}},
every y tick/.append style={white!15!black},
ymin=-0.8,
ymax=0.8,
ylabel style={font=\color{white!15!black}},
ylabel={Eje imaginario},
xlabel style={font=\color{white!15!black}},
xlabel={Eje real},
axis background/.style={fill=white},
legend cell align=left,
restrict y to domain=-20:20
]
\addplot[only marks, mark=+, mark options={}, mark size=2.5000pt, draw=red, forget plot] table[]{Diagramas/data/nyquist_g3-1.tsv};
\addplot [color=gray, dotted, forget plot]
  table[]{Diagramas/data/nyquist_g3-2.tsv};
\addplot [color=gray, dotted, forget plot]
  table[]{Diagramas/data/nyquist_g3-3.tsv};
\draw[dashdotted, draw=gray] (axis cs:0,0) circle [radius=1];
\addplot [color=gray, dashdotted, forget plot]
  table[]{Diagramas/data/nyquist_g3-4.tsv};
\addplot [color=gray, dashdotted, forget plot]
  table[]{Diagramas/data/nyquist_g3-5.tsv};

\addplot[area legend, draw=mycolor2, fill=mycolor2, forget plot]
table[] {Diagramas/data/nyquist_g3-6.tsv}--cycle;

\addplot[area legend, draw=mycolor2, fill=mycolor2, forget plot]
table[] {Diagramas/data/nyquist_g3-7.tsv}--cycle;
\addplot [color=mycolor2, forget plot]
  table[]{Diagramas/data/nyquist_g3-8.tsv};
\addplot[only marks, mark=*, mark options={}, mark size=1.5000pt, draw=mycolor1, fill=mycolor1] table[]{Diagramas/data/nyquist_g3-9.tsv};
  \addlegendentry{Margen de fase}
\addplot[only marks, mark=*, mark options={}, mark size=1.5000pt, draw=mycolor4, fill=mycolor4] table[]{Diagramas/data/nyquist_g3-10.tsv};
  \addlegendentry{Margen de ganancia}
\end{axis}
\end{tikzpicture}%

  \caption{Diagrama de Nyquist del sistema con retardo crítico}
  \label{fig:nyquist-g3}
\end{figure}

\begin{figure}[h]
  \centering
  % This file was created by matlab2tikz.
%
%The latest updates can be retrieved from
%  http://www.mathworks.com/matlabcentral/fileexchange/22022-matlab2tikz-matlab2tikz
%where you can also make suggestions and rate matlab2tikz.
%
\begin{tikzpicture}

\begin{axis}[%
width=9.632cm,
height=2.765cm,
at={(0cm,0cm)},
scale only axis,
unbounded coords=jump,
separate axis lines,
every outer x axis line/.append style={white!15!black},
every x tick label/.append style={font=\color{white!15!black}},
every x tick/.append style={white!15!black},
xmode=log,
xmin=0.1,
xmax=15.707963267949,
xminorticks=true,
every outer y axis line/.append style={white!15!black},
every y tick label/.append style={font=\color{white!15!black}},
every y tick/.append style={white!15!black},
ymin=-1080,
ymax=5,
ytick={-1080, -900, -720, -540, -360, -180,  0},
ylabel style={font=\color{white!15!black}},
ylabel={Fase $[\unit{deg}]$},
axis background/.style={fill=white}
]
\addplot [color=gray, line width=1.5pt, forget plot]
  table[]{Diagramas/data/bode_g3-1.tsv};
\addplot [color=gray, line width=1.5pt, forget plot]
  table[]{Diagramas/data/bode_g3-2.tsv};
\addplot [color=gray, dashdotted, forget plot]
  table[]{Diagramas/data/bode_g3-3.tsv};
\addplot [color=gray, dashdotted, forget plot]
  table[]{Diagramas/data/bode_g3-4.tsv};
\addplot [color=mycolor2, forget plot]
  table[]{Diagramas/data/bode_g3-5.tsv};
\addplot[only marks, mark=*, mark options={}, mark size=1.5000pt, draw=mycolor2, fill=white, forget plot] table[]{Diagramas/data/bode_g3-6.tsv};
\addplot[only marks, mark=*, mark options={}, mark size=1.5000pt, draw=mycolor2, fill=mycolor2, forget plot] table[]{Diagramas/data/bode_g3-7.tsv};
\addplot [color=gray, dotted, forget plot]
  table[]{Diagramas/data/bode_g3-8.tsv};
\addplot [color=gray, dotted, forget plot]
  table[]{Diagramas/data/bode_g3-9.tsv};
\addplot [color=gray, line width=1.5pt, forget plot]
  table[]{Diagramas/data/bode_g3-10.tsv};
\addplot [color=gray, line width=1.5pt, forget plot]
  table[]{Diagramas/data/bode_g3-11.tsv};
\addplot [color=gray, line width=1.5pt, forget plot]
  table[]{Diagramas/data/bode_g3-12.tsv};
\addplot [color=gray, line width=1.5pt, forget plot]
  table[]{Diagramas/data/bode_g3-13.tsv};
\addplot [color=gray, dashdotted, forget plot]
  table[]{Diagramas/data/bode_g3-14.tsv};
\addplot [color=gray, dotted, forget plot]
  table[]{Diagramas/data/bode_g3-15.tsv};
\addplot [color=gray, dotted, forget plot]
  table[]{Diagramas/data/bode_g3-16.tsv};
\addplot [color=gray, line width=1.5pt, forget plot]
  table[]{Diagramas/data/bode_g3-17.tsv};
\addplot [color=gray, line width=1.5pt, forget plot]
  table[]{Diagramas/data/bode_g3-18.tsv};
\addplot [color=gray, line width=1.5pt, forget plot]
  table[]{Diagramas/data/bode_g3-19.tsv};
\addplot [color=gray, line width=1.5pt, forget plot]
  table[]{Diagramas/data/bode_g3-20.tsv};
\addplot [color=gray, dashdotted, forget plot]
  table[]{Diagramas/data/bode_g3-21.tsv};
\addplot [color=gray, dotted, forget plot]
  table[]{Diagramas/data/bode_g3-22.tsv};
\addplot [color=gray, dotted, forget plot]
  table[]{Diagramas/data/bode_g3-23.tsv};
\end{axis}

\begin{axis}[%
width=9.632cm,
height=2.765cm,
at={(0cm,3.235cm)},
scale only axis,
unbounded coords=jump,
separate axis lines,
every outer x axis line/.append style={white!15!black},
every x tick label/.append style={font=\color{white!15!black}},
every x tick/.append style={white!15!black},
xmode=log,
xmin=0.1,
xmax=15.707963267949,
xtick={0.1,1,10},
xticklabels={\empty},
xminorticks=true,
every outer y axis line/.append style={white!15!black},
every y tick label/.append style={font=\color{white!15!black}},
every y tick/.append style={white!15!black},
ymin=-50,
ymax=-0,
ylabel style={font=\color{white!15!black}},
ylabel={Magnitud $[\unit{dB}]$},
axis background/.style={fill=white},
title style={font=\color{white!15!black}},
]
\addplot [color=gray, line width=1.5pt, forget plot]
  table[]{Diagramas/data/bode_g3-24.tsv};
\addplot [color=gray, line width=1.5pt, forget plot]
  table[]{Diagramas/data/bode_g3-25.tsv};
\addplot [color=gray, dashdotted, forget plot]
  table[]{Diagramas/data/bode_g3-26.tsv};
\addplot [color=gray, dashdotted, forget plot]
  table[]{Diagramas/data/bode_g3-27.tsv};
\addplot [color=mycolor2, forget plot]
  table[]{Diagramas/data/bode_g3-28.tsv};
\addplot[only marks, mark=*, mark options={}, mark size=1.5000pt, draw=mycolor2, fill=white, forget plot] table[]{Diagramas/data/bode_g3-29.tsv};
\addplot[only marks, mark=*, mark options={}, mark size=1.5000pt, draw=mycolor2, fill=mycolor2, forget plot] table[]{Diagramas/data/bode_g3-30.tsv};
\end{axis}
\end{tikzpicture}%

  \caption{Diagrama de Bode del sistema con retardo crítico}
  \label{fig:bode-g3}
\end{figure}

\FloatBarrier
\subsection{Comentarios}

\textbf{NO OLVIDAR!!!}


%%% G %%%
\begin{itemize}
  \item Idem (a), (b), (c), (d) y (e) comparando con un controlador proporcional discreto con kc = 20 incluyendo retardo.
\end{itemize}

