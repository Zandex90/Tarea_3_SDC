\section{Pregunta \texttt{c)}}\label{pregunta-c}

\subsection{Desarrollo}

Luego, se desea determinar a partir de los resultados obtenidos anteriormente,
los valroes de ganancia crítica $\mora{k_{cr}}$ y de retardo crítico $\nara{t_{cr}}$.
Para la ganancia crítica, puesto que el margen de ganancia obtenido representa
la ganancia máxima que se le puede aplicar al sistema con controlador $\mora{k_{c}} = 10\unit{m}$,
entonces, la ganancia máxima del sistema general, la ganancia crítica, sería:
\begin{equation}
  \mora{k_{cr}} = \mora{k_{c}} \cdot \te{M.G.}
\end{equation}

En cuanto al retardo crítico, este corresponde al máximo retardo que se le
puede aplicar al sistema antes de que se vuelva marginalmente estable. Además,
el margen de fase representa cuánto se puede rotar el Nyquist de un sistema
hasta que sea marginalmente estable. Entonces, como rotar el Nyquist se consigue
aplicando un retardo, entonces se ve de esto que el retardo crítico es:
\begin{equation}
  \mora{t_{cr}} = \frac{\te{M.F.}\ \te{(rad)}}{\omega_{g}}
\end{equation}

Con $\omega_{g}$ la frecuencia de cruce de ganancia, la cual para el sistema
estudiado es $\omega_{g} = 10.4\ \unit{rad/s}$.

Entonces, la ganancia crítica y el retardo crítico del sistema son,
\begin{equation}
  \boxed{\mora{k_{cr}} = 22.247\unit{m}} \qquad \boxed{\mora{t_{cr}} = 16.75\ \unit{ms}}
\end{equation}


\FloatBarrier
\subsection{Comentarios}

\textbf{NO OLVIDAR!!!}


%%% C %%%
\begin{itemize}
  \item Relación entre el MG y la ganancia crítica del sistema.
  \item Relación entre el MF y retardo crítico del sistema.
  \item Qué parámetro del sistema define la ganancia crítica.
  \item Qué parámetro del sistema define el retardo crítico.
\end{itemize}
