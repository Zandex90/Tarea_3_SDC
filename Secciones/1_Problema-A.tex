\section{Pregunta \texttt{a)}}\label{pregunta-a}

\subsection{Desarrollo}

En este apartado nos piden calcular, el margen de ganancia (M.G.) y margen de fase
(M.F.). Para esto, vamos a utilzar la función de transferencia del lazo directo del
sistema. De la tarea anterior \cite{tarea-2-sdc} sabemos que esta es:
\begin{equation}
  LD(s) = \mora{k_c} \cdot \nara{k_a} \cdot \frac{\nara{b_{0\omega}} \left(\nara{b_{1\psi}}s + \nara{b_{0\psi}}\right)}
  {\left(\nara{a_{1\omega}}s + \nara{a_{0\omega}}\right)\left(\nara{a_{2\psi}}s^2 + \nara{a_{1\psi}}s + \nara{a_{0\psi}}\right)} \cdot \nara{k_{st}}
\end{equation}

Luego, para determinar los valores pedidos, vamos a utilizar las herramientas
de estudio de estabilidad de sistemas disponibles en \textit{MATLAB}. En
específico, vamos a utilizar el comando \texttt{margin}. Este comando nos
retornará los valores de margen de fase y margen de ganancia en grados y en
ganancia absoluta respectivamente. Entonces, se tiene que estos son:

\begin{equation}
  \boxed{\textbf{M.G.} = 2.2247 = 6.945\ \unit{dB}} \qquad \boxed{\textbf{M.F.} = \ang{9.9833}}
\end{equation}

El código utilizado para responder esta pregunta puede observarse en el
\autoref{lst:Cod_problema_A}.

\subsection{Comentarios}

El margen de ganancia de $6.945$ dB (equivalente a una ganancia absoluta de $2.2247$) existe porque, en la frecuencia donde la fase total del sistema es de $-180^\circ$, la magnitud de la función de transferencia es menor que uno (o $0$ dB). Esto significa que el sistema no alcanza la ganancia crítica que provocaría la inestabilidad, gracias a los parámetros y constantes presentes en la función de transferencia, como las ganancias $\mora{k_c}$, $\nara{k_a}$ y $\nara{k_{st}}$, y los coeficientes de los polinomios en $s$.

Un M.F cercano a $10^\circ$ es relativamente pequeño y sugiere que el sistema está próximo al límite de estabilidad. En la práctica, esto significa que el sistema puede ser sensible a perturbaciones, variaciones en los parámetros del sistema o retardos no considerados en el modelo. Un margen de fase reducido puede conducir a respuestas transitorias oscilatorias y a una menor tolerancia a cambios no previstos, por lo que sería recomendable ajustar el controlador o los parámetros del sistema para aumentar este margen.



