\section{Pregunta \texttt{a)}}\label{pregunta-a}

\subsection{Desarrollo}

En este apartado nos piden calcular, el margen de ganancia (M.G.) y margen de fase (M.F.) , que lo realizaríamos con el de el lazo directo (L.D), pero por la tarea 1\cite{tarea-1-sdc} sabemos que el L.D es  \( \mora{k_c} \nara{ k_a} \frac{\rojo{\psi}(s)}{\verd{v_i}(s)}\nara{k_st} \) que en desarrollando es:


\begin{equation}
  LD(s) = \mora{k_c} \cdot \nara{k_a} \cdot \frac{\nara{b_{0\omega}} \left(\nara{b_{1\psi}}s + \nara{b_{0\psi}}\right)}
  {\left(\nara{a_{1\omega}}s + \nara{a_{0\omega}}\right)\left(\nara{a_{2\psi}}s^2 + \nara{a_{1\psi}}s + \nara{a_{0\psi}}\right)} \cdot \nara{k_{st}}
\end{equation}

%%%%%%  ESTO NO SE SI DEJARLO %%%%
Donde si reemplazamos numéricamente lo anterior tenemos: 
\begin{equation}
  LD(s) = 10m \cdot 100 \cdot \frac{0.5294 \, s + 2.362}{s^3 + 2.835 \, s^2 + 80.87 \, s + 119.5} \cdot \frac{180^{\circ}}{\pi}
\end{equation}
%%%%

Ahora bien daremos uso a la herramienta de \textit{MATLAB}, utilizando el comando \verb|margin|, ya que, este comando nos devolvería el M.G. y M.F. directamente. Solo que hay que tener presente que la M.F. nos la devuelve en "Grados", y el M.G. no lo devuelve en la cantidad de veces que puedo multiplicar la ganancia antes que se haga inestable, ahora bien si quisiéramos en M.G. en $dB$ solo tendríamos que hacer la relación de \(dB = 20log(M.G.)\) siendo M.G. el valor en la cantidad de veces, ahora bien para el problema \hyperref[pregunta-a]{\texttt{a)}}  tenemos que:

\vspace*{0.25cm}

\fbox{
  \begin{minipage}{0.9\linewidth}
    \begin{itemize}
      \item \textbf{M.G.=} $2.2247$  (que es igual a $6.945 \unit{dB}$)
      \item \textbf{M.F.=} $ 9.9833^{\circ}$
    \end{itemize}
  \end{minipage}
  }

\vspace*{0.5cm}
El código utilizado para responder esta pregunta en el \autoref{lst:Cod_problema_A}.







\subsection{Comentarios}
\textbf{NO OLVIDAR!!!}

%%% A %%%
\begin{itemize}
  \item Si existe el M.F. qué característica o parámetro es el responsable.
  \item Si no existe el M.F. qué característica o parámetro es el responsable.
  \item Si existe el M.G. qué característica o parámetro es el responsable.
  \item Si no existe el M.G. qué característica o parámetro es el responsable.
  \item Implicancias prácticas del valor de M.F. obtenido.
  \item Implicancias prácticas del valor de M.G. obtenido.
\end{itemize}



