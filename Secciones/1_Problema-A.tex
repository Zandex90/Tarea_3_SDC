\section{Pregunta \texttt{a)}}\label{pregunta-a}

\subsection{Desarrollo}

En este apartado nos piden calcular, el margen de ganancia (M.G.) y margen de fase
(M.F.). Para esto, vamos a utilzar la función de transferencia del lazo directo del
sistema. De la tarea anterior \cite{tarea-2-sdc} sabemos que esta es:
\begin{equation}
  LD(s) = \mora{k_c} \cdot \nara{k_a} \cdot \frac{\nara{b_{0\omega}} \left(\nara{b_{1\psi}}s + \nara{b_{0\psi}}\right)}
  {\left(\nara{a_{1\omega}}s + \nara{a_{0\omega}}\right)\left(\nara{a_{2\psi}}s^2 + \nara{a_{1\psi}}s + \nara{a_{0\psi}}\right)} \cdot \nara{k_{st}}
\end{equation}

Luego, para determinar los valores pedidos, vamos a utilizar las herramientas
de estudio de estabilidad de sistemas disponibles en \textit{MATLAB}. En
específico, vamos a utilizar el comando \texttt{margin}. Este comando nos
retornará los valores de margen de fase y margen de ganancia en grados y en
ganancia absoluta respectivamente. Entonces, se tiene que estos son:

\begin{equation}
  \boxed{\textbf{M.G.} = 2.2247 = 6.945\ \unit{dB}} \qquad \boxed{\textbf{M.F.} = \ang{9.9833}}
\end{equation}

El código utilizado para responder esta pregunta puede observarse en el
\autoref{lst:Cod_problema_A}.

\subsection{Comentarios}
\textbf{NO OLVIDAR!!!}

%%% A %%%
\begin{itemize}
  \item Si existe el M.F. qué característica o parámetro es el responsable.
  \item Si no existe el M.F. qué característica o parámetro es el responsable.
  \item Si existe el M.G. qué característica o parámetro es el responsable.
  \item Si no existe el M.G. qué característica o parámetro es el responsable.
  \item Implicancias prácticas del valor de M.F. obtenido.
  \item Implicancias prácticas del valor de M.G. obtenido.
\end{itemize}



